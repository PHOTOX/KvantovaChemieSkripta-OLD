V kapitole \textbf{XX} jsme se seznámili s různými metodami, kterými můžeme řešit Schr\"dingerovu rovnici. Mohli jsem tak nabýt dojmu, že kupříkladu metoda úplné konfigurační interakce představuje svatý grál kvantové chemie, který nás dovede rovnou k "pravdě". Ve skutečnosti ani energie a vlnová funkce vypočítaná metodou konfigurační interakce s nekonečnou jednoelektronovou bází neposkytne přesnou hodnotu energie. Základním problémem je zanedbání relativistických efektů. Mluvíme-li v této kapitole o teorii relativity, máme tím na mysli speciální teorii relativity, která nás zpravuje o mechanice částic pohybujících se rychlostí blízkou rychlosti světla. Dozvídáme se z ní kupříkladu, že základní parametr každé hmotné částice, totiž její hmotnost $m$, závisí na rychlosti pohybu této částice:

\begin{equation}
\label{rov:XXX}
\end{equation}

\noindent kde $m_0$ je klidová hmotnost částice, $v$ její rychlost a  $c$ je rychlost světla ve vakuu. Měli bychom se v chemii vůbec trápit relativistickými efekty? Odpověď nám naznačí následující příklad.

\textbf{Příklad}

Odhadněte hmotnost $m$ elektronu v atomu vodíku a v 1s orbitalu atomu rtuti.

Řešení: K odhadu použijeme výraz pro rychlost z Bohrovy teorie atomů vodíkového typu

\begin{equation}
\label{rov:XXX}
\end{equation}

\noindent kde $Z$ je nábojové číslo jádra, $n$ je hlavní kvantové číslo, $c$ je rychlost světla ve vakuu a $\alpha$ je tzv. konstanta jemné struktury mající hodnotu mající hodnotu $1/137$. V případě rtuti jde jen o hrubý odhad, jádro rtuti bude stíněno i zbylými ostatními elektrony. V případě 1s orbitalu půjde ale o stínění nevýznamné. Pro vodík tak dostáváme rychlost

\begin{equation}
\label{rov:XXX}
\end{equation}

\noindent a pro rtuť s $Z=80$

\begin{equation}
\label{rov:XXX}
\end{equation}

\noindent Takže zatímco u atomu vodíku se hmotnost téměř nezmění, u těžších atomů už musíme brát relativistické efekty nejspíše vážně. Změna hmotnosti elektronu totiž jistě povede k odlišnému chování jednotlivých orbitalů.

Pro popis relativistických efektů je nutné vytvořit rovnici, která je v souladu jak s postuláty kvantové mechaniky, tak s teorií relativity. V klasické, nerelativistické mechanice je energie volné částice dána jako

\begin{equation}
\label{rov:XXX}
\end{equation}
  
  
\noindent čemuž odpovídá hamiltonián 

\begin{equation}
\label{rov:XXX}
\end{equation}

\noindent a časově-závislá Schr\"odingerova rovnice 

\begin{equation}
\label{rov:XXX}
\end{equation}

Mohli bychom být v pokušení vyjít nyní s relativistického výrazu pro energii

\begin{equation}
\label{rov:XXX}
\end{equation}

\noindent a s použitím stejných pravidel ($E$ nahradíme XXXXXXXXXXXXXXXXXXXXXXXX a $p$ XXXXXXX) bychom získali tzv. Kleinovu-Gordonovu rovnici

\begin{equation}
\label{rov:XXX}
\end{equation}

Bohužel se ukazuje, že v této rovnici se nezachovává počet částic. Není ji možné použít pro popis fermionů (hodí se však pro popis bosonů). Správnou rovnici pro elektron formuloval Paul Dirac. Vyšel opět z rovnice \textbf{XX} a svou geniální intuicí dospěl k závěru, že elektron je popsán nikoliv jednou vlnovou funkcí, ale rovnou uspořádanou čtveřicí vlnových funkcí:

\begin{equation}
\label{rov:XXX}
\end{equation}

\noindent které se pak řídí Diracovou rovnicí formálně připomínající rovnici Schr\"odingerovu:

\begin{equation}
\label{rov:XXX}
\end{equation}

\noindent kde $\alpha$ a $\beta$ představují matice rozměru 4x4. \textbf{Nedat je sem explicitne? Je mozne predelat 4x4?}  Jednotlivé komponenty vlnové funkce odpovídají elektronu se spinem $\alpha$ a $\beta$ a pozitronu se spinem $\alpha$ a $\beta$. Dirac tak ukázal, že požadavek na kompatibilitu mezi kvantovou mechanikou a teorií relativity vede automaticky k elektronovému spinu a také k existenci antičástic.  
    
  

\subsection{Relativistické efekty v chemii}

Relativistické efekty můžeme rozdělit do dvou základních skupin. 

\begin{itemize} 

\item Skalární relativistické efekty. Jde o efekty spojené s rozdílným relativistickým výrazem pro energii a tedy s rozdílnou hmotností elektronu. Nejde tedy o jevy, které by souvisely s tím, že vlnová funkce má čtyři komponenty. 

Elektron má díky relativitě větší hmotnost. Můžeme tak očekávat, že se díky tomu elektron bude chovat klasičtěji, bude se pohybovat blíže k atomovému jádru. V případě s a p orbitalů tak bude docházet ke kontrakci orbitalů  \footnote{Tento pojem nelze zaměňovat s relativistickou kontrakcí délek.} Díky kontrakci s a p orbitalů je ale na druhou stranu jádro lépe stíněno. Elektrony v d a f orbitalech proto pociťují menší efektivní náboj a díky tomu dochází k jejich expanzi. Tyto jevy mají rozličné projevy v chemii. Například v případě zlata dochází ke stabilizaci elektronů v 6s orbitalu a k ionizaci dochází z 5d orbitalů. Vznikají tak trivalentní nebo pentavalentní ionty zlata. Díky relativistickým efektům jsou také stabilnější vyšší oxidační stavy kovů, takže může existovat kupříkladu i ion IrO$_4^+$, který obsahuje iridium v formálním oxidačním stavu +IX. Dochází také k určitému zkrácení vazebné délky, částečně relativistickým efektem je i lanthanoidová kontrakce. Jedině díky relativitě můžeme využívat v automobilech olověných akumulátorů. Olovo má elektronovou konfiguraci 6s$^2$6p$^2$. Elektrony v těchto orbitalech jsou relativisticky stabilizované, díky čemuž má olovo v oxidačním stupni +IV vyšší energii. To však v důsledku vede k větší změně Gibbsovy energie v elektrodové reakci odehrávající se v akumulátorech:  

\begin{equation}
\label{rov:XXX}
\end{equation}

Bez zahrnutí relativistických efektů by zlato nebylo žluté a rtuť by nebyla kapalná. S relativitou se zkrátka v chemii potkáváme docela často. 

\item Relativistické efekty spojené se spinem. Zde máme na mysli především tzv. spin-orbitální interakci, o které již byla řeč v kapitole \textbf{XXX}. Tento efekt se projeví korekcí k nerelativistickému hamiltoniánu

\begin{equation}
\label{rov:XXX}
\end{equation}

\noindent
kde $\vec{L}$ je orbitální moment a $\vec{S}$ je spinový moment elektronu. Spin-orbitální interakce se opět uplatňuje zejména pro těžší atomy. Způsobuje rozštěpení mezi energetickými hladinami, pro těžší prvky často značné.

\end{itemize}

\subsection{Kvantově-chemické výpočty relativistických efektů}

Při výpočtech zahrnujících efekty spojené s teorií relativity bychom mohli vyjí z Diracovy rovnice. Ta ovšem popisuje pohyb pouze jednoho elektronu. Je proto třeba přidat člen popisující interakci mezi elektrony. Mohli bychom k Diracovu členu přidat coulombické odpuzování mezi elektrony. Takovýto přístup ale není relativisticky plně konzistentní. Mimo jiné předpokládáme, že k interakci mezi elektrony dochází okamžitě, což při konečné rychlosti světla není pravda. Lepším přístupem je konstrukce tzv. Diracova-Coulombova-Breitova hamiltoniánu

\begin{equation}
\label{rov:XXX}
\end{equation}

\noindent kde \textbf{XXX} je jednočásticový Diracův hamiltoninán

\begin{equation}
\label{rov:XXX}
\end{equation}


\noindent a dvou-částicový člen je dán jako součet coulombického a Breitova členu


\begin{equation}
\label{rov:XXX}
\end{equation}

Rovnici tohoto typu pro čtyř-komponentovou vlnovou funkci můžeme dále zjednodušovat. Je snadné se zbavit dvou komponent popisujících pozitrony. Často se také používají ryze rovnice popisující pouze skalární relativistické efekty, například v rámci přístupu ZORA (a angl. \textit{Zero Order Relativistic Approximation}).

Pragmatickou cestou k zahrnutí relativistických efektů je použití tzv. efektivních relativistických pseudopotenciálů pro vnitřní elektrony (ECP, z angl. \textit{Effective Core Pseudopotential}). Relativistické příspěvky jsou totiž významné zejména pro elektrony vnitřních slupek, valenční elektrony se pohybují daleko pomaleji, neboť jejich interakce s atomovými jádry je již hodně stíněna právě vnitřními elektrony. Chemika ale vnitřní elektrony obvykle mnoho nezajímají, chemické reakce představují děje ve valenční sféře. Můžeme proto vnitřní elektrony nahradit vhodně zvoleným potenciálem, který simuluje vnitřní elektrony. Tento potenciál si jednou pro vždy nastavíme pro daný atom s pomocí plně relativistického výpočtu a pak jej můžeme volně použít pro libovolné molekuly, jejíž je daný atom součástí. Získáme tak kvalitnější výsledek a navíc jako bonus je výpočet časově méně náročný, neboť vnitřní elektrony již do výpočtů nezahrnujeme.         
  
Oblast relativistické kvantové chemie představuje intenzivní předmět současného výzkumu a není v možnostech tohoto textu se této otázce do detailu věnovat. Zájemce můžeme odkázat na dva výtečné přehledné články pod Pekky Pyyk\"oho. P. Pykk\"o, \textit{Relativistic Effects in Chemistry: More Common Than You Thought.} Annu. Rev. Phys. Chem. 2012, 63, 45-64 a P. Pykk\"o, JP Desclaux, \textit{Relativity and the Periodic Systems of Elements} Acc. Chem. Res. 1979, 51, 276-281. 



      

   




 




      

      



