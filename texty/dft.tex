Pomocí metod založených na teorii funkcionálu hustoty(dále jen DFT metod) se v posledních zhruba dvaceti let provádí většina výpočtů elektronové struktury. Popularita DFT metod je dána jejich přijatelnou výpočetní náročností, která o mnoho nepřevyšuje Hartreeho-Fockovu 
metodu. Výpočty jsou ale typicky daleko přesnější, neboť DFT metody zahrnují korelační energii. Efektivita DFT je dána tím, že nepracuje s poměrně složitou vlnovou funkcí (tj. funkcí 3N souřadnic elektronů), ale s tzv. elektronovou hustotou, což je funkce pouze tří prostorových souřadnic. 

Představme si základní pojmy. Začněme pojmem \textbf{funkcionál}. Pojďme si nejprve připomenout, jak funguje funkce. Do funkce vložíme nezávislé proměnou, tedy nějaké číslo, a na oplátku dostaneme číslo jiné, neboli závisle proměnnou. Jde tedy o zobrazení z prostoru (kupř. reálných) čísel opět do prostoru reálných čísel. U funkcionálu je to velmi podobné, akorát na vstupu není číslo, ale funkce. Pokud to tedy řekneme více matematicky, funkcionál je zobrazení z prostoru funkcí na prostor(kupř. reálných) čísel.

Takovým jednoduchým funkcionálem je určitý integrál
$$
F[f(x)] = \int_a^b f(x) \mathrm{d}x, 
$$
Určitý integrál potřebuje dodat vstupní funkci a po jeho vyčíslení dostaneme jedno jediné číslo. Povšimněte si zde zápisu funkcionálu pomocí hranatých závorek.  
Pro složitější příklad funkcionálu nemusíme chodit daleko, stačí se podívat, jak vypadá funkcionál energie v kvantové mechanice.
$$
E[\psi] = \int \psi^*\hat{H}\psi \mathrm{d}\tau .
$$
S některými funkcionály jsme se tak již setkali.

Mezi funkcemi a funkcionály existují i další podobnosti. Pojmy jako minimum a maximum funkcionálu mají prakticky stejný význam. Existuje i funkcionální analogie k dobře známé derivaci funkce --- mluvíme o \textbf{variaci funkcionálu}.
Když děláme derivaci funkce, tak se vlastně díváme, co se děje se závisle proměnnou, když trochu změníme nezávisle proměnnou.
U variace je to podobné. Zajímá nás, jak se změní hodnota funkcionálu, když mírně změníme naši funkci. Přesná definice je složitější a neuvádíme ji zde. Bude ale pro nás důležité, že pro variace platí podobné vztahy, jako pro derivace. Dá se například ukázat, že v minimu funkcionálu je jeho variace nulová. Již dobře známý variační princip (konečně víme, proč se mu tak říká!) pak můžeme napsat jednoduše pomocí variace tohoto funkcionálu energie jako
$$
\delta E[\Psi] = 0 .
$$

Nyní se můžeme vrátit k definici ústřední veličiny této kapitoly --- elektronové hustoty $\rho(\mathbf{r})$.
Elektronová hustota má na rozdíl od vlnové funkce přímý fyzikální význam. Jedná se o pravděpodobnost, že v nějakém bodě prostoru najdeme \textbf{nějaký elektron}. Je důležité si uvědomit rozdíl mezi elektronovou hustotou a čtvercem vlnové funkce, který taktéž udává hustotu pravděpodobnosti. Čtverec vlnové funkce nám udává pravděpodobnost, že první elektron má spin $m_{s1}$ a je v bodě $\mathbf{r}_1$, druhý elektron má spin $m_{s2}$ a je v bodě ($\mathbf{r}_2$) atd. Jedná se tedy o mnohem složitější veličinu, která závisí na celkem $4N$ proměnných, zatímco elektronová hustota závisí jen na třech proměnných.

Elektronová hustota souvisí s vlnovou funkcí systému dle vztahu
\begin{equation}
\rho=N \int |\psi(\textbf{r}_1,\textbf{r}_2,...,\textbf{r}_n)|^2 \mathrm{d}\textbf{r}_2\dots\mathrm{d}\textbf{r}_n .
\end{equation}
\textbf{Chce to okomentovat, proc tomu tak je} Z této definice hned plyne několik důležitých vlastností.

\begin{itemize}
\item Elektronová hustota je nezáporná veličina, platí tedy
\begin{equation}
\rho(\mathbf{r})  > 0 .
\end{equation}
\item Pokud zintegrujeme elektronovou hustotu přes celý prostor, dostaneme počet elektronů v systému jako
\begin{equation}
\int \rho\mathrm{d}r = N
\end{equation}

\item V poloze jader má elektronová hustota maxima.
\item Pro tato maxima platí (XXX teorém), že
\begin{equation}
\lim_{r_i \to 0} \left[ \frac{\delta}{\delta r}+2Z_A\right]\hat{\rho(r)}=0, 
\end{equation}
kde $\hat{\rho}$ je angulárně zprůměrovaná hodnota elektronové hustoty a $Z_A$ je náboj příslušného atomového jádra.
\end{itemize}

Z těchto vlastností vidíme, že pokud známe elektronovou hustotu, tak zároveň také můžeme zjistit počet elektronů, polohu jader i jejich náboj. To jsou ale přesně ty informace, které jsou potřeba ke specifikaci molekulárního hamiltoniánu
\begin{equation}
\hat{H}=\sum_{i=1}^N -\frac{1}{2}\Delta_i+\sum_{i=1}^N\sum_{j=i+1}^N\frac{1}{r_{ij}}-\sum_{i=1}^N\sum_{A=1}^K \frac{Z_A}{r_{iA}} ,
\label{rov:ham_dft}
\end{equation}
kde jsme vynechali pro jednoduchost člen popisující odpuzování jader, jenž je stejně v rámci Bornovy--Oppenheimerovy aproximace roven konstantě. Pokud ale elektronová hustota jednoznačně určuje hamiltonián, tak poté z řešení SCHR taky energii a vlnovou funkci, a tudíž všechny potřebné veličiny. Podobným způsobem se zřejmě ubíraly úvahy Hohenberga a Kohna, kteří postavili teorii DFT na pevné fyzikální základy.

\subsection{Hohenbergovy--Kohnovy teorémy}

V roce 1964 publikovali Hohenberg s Kohnem slavný článek, který odstartoval vývoj DFT metod.
V tomto článku byly dokázány dva důležité teorémy. K jejich lepšímu pochopení si ještě musíme definovat pojem tzv. externího potenciálu $\nu_{ext}$. Ten má pro případ hamiltoniánu \eqref{rov:ham_dft} tvar
\begin{equation}
\nu_{ext} = \sum_{i=1}^N\sum_{A=1}^K \frac{Z_A}{r_{iA}} .
\end{equation} 
Jedná se tedy o celkovou interakci elektronů s coulombickým potenciálem atomových jader. Jelikož se v DFT na vše díváme z pohledu elektronů, tak se tomu potenciálu říká externí, jelikož nepochází ze samotných elektronů. Je důležité si uvědomit, že externí potenciál vlastně definuje hamiltonián \ref{rov:ham_dft}, neboť ostatní členy triviálně závisejí pouze na počtu elektronů v systému. Nyní již můžeme přejít ke slíbeným teorémům.

První Hohenbergův--Kohnův teorém je existenční. Zní takto:
\uv{\textbf{Pro libovolný systém interagujících elektronů je externí potenciál $\nu_{ext}$} jednoznačně určen elektronovou hustotou (až na konstantu) }

Důsledek tohoto tvrzení jsme si již naznačili dříve. Pokud máme jednoznačně daný externí potenciál, pak také známe hamiltonián a tudíž můžeme v principu spočítat vlnovou funkci i energii, které jsou tudíž jednoznačně určeny pouze elektronovou hustotou. \textbf{Elektronová energie je tedy jednoznačným funkcionálem elektronové hustoty} neboli
\begin{equation}
E_{el}=E_{el}[\rho] .
\end{equation}

Pro důkaz tohoto tvrzení si musíme ještě odvodit pomocn vztah pro integraci externího potenciálu vynásobeného vlnovou funkcí:

Důkaz tohoto teorému se provádí sporem. Předpokládejme, že jedné dané hustotě $\rho$ přísluší dva různé externí potenciály $\nu_{ext1}$ a $\nu_{ext2}$, kterým přísluší hamiltoniány vlnové funkce $\psi_1$ a $\psi_2$  Pak díky variačnímu principu musí platit

\begin{equation}
E_1 = \int \psi_1^* \hat{H}_1 \psi_1 \mathrm{d}\tau < \int \psi_2^* \hat{H}_1 \psi_2 . \mathrm{d}\tau
\end{equation}
Když do hamiltoniánu $\hat{H_1}$ přiřadíme funkci $\Psi_1$, která ji nepřísluší, tak musíme dostat větší energii. Nyní chytře přepíšeme pravou stranu nerovnosti jako  
\begin{equation}
\int \psi_2^* \hat{H}_1 \psi_2 = \int \psi_2^* \hat{H}_2 \psi_2\mathrm{d}\tau  + \int \psi_2 \left[\hat{H}_1-\hat{H}_2\right] \psi_2\mathrm{d}\tau = E_2 + \int \rho(\mathbf{r})(\nu_1-\nu_2)\mathrm{d}\mathbf{r}  
\end{equation}
V poslední rovnosti jsme využili toho, že oba hamiltoniány se liší pouze externím potenciálem.

Vyšlo nám tedy, že
\begin{equation}
E_1 < E_2+\int \rho(\mathbf{r})(\nu_1-\nu_2)\mathrm{d}\mathbf{r}
\label{rov:HK1_1}
\end{equation}
Tu samou argumentaci bychom ale mohli také aplikovat opačně, a dostali bychom
\begin{equation}
E_2 < E_1+\int \rho(\mathbf{r})(\nu_1-\nu_2)\mathrm{d}\mathbf{r} .
\label{rov:HK1_2}
\end{equation}
Sečtením obou rovnic \eqref{rov:HK1_1} a \eqref{rov:HK1_2} tedy dostáváme
\begin{equation}

\end{equation}

\item 2. HK teorém 
Předpokládejme, že pro daný externí potenciál $\nu_{ext,0}$ je správná elektronová hustota $\rho_0$. Pak pro jakoukoli funkci $\rho$ bude platit: $$ E[\rho_0] < E[\rho] $$

Důkaz:
Z prvního HK teorému plyne, že každá funkce $\rho$ jednoznačně definuje externí potenciál (který je odlišný od $v_{ext,0}$), a tudíž i nějakou vlnovou funkci $\psi$. Pokud ale pro tuto vlnovou funkci vyčíslíme energii, pak nám z již známého variačního principu plyne:
\begin{equation}
E^\prime=\int \psi^{\prime *} \hat{H} \psi^{\prime} \mathrm{d}\tau = E[\rho] > E_0 = E_0 [\rho_0] ,
\end{equation}
což jsme chtěli dokázat.
\end{itemize}


Formálně můžeme funkcionál energie napsat jako
\begin{equation}
E=\int \psi^*\hat{H}\psi \mathrm{d}\tau = \int \psi^*\hat{T}\psi\mathrm{d}\tau \int \psi^*\hat{V_{el}}\psi\mathrm{d}\tau \int \psi^*\nu_{ext}\psi\mathrm{d}\tau=\hat{T}[\rho]+\hat{V}_{ee}[\rho]+\hat{V}_{ne}[\rho] .
\end{equation}
Funkcionál energie jsme tedy rozdělili na funkcionál kinetické energie elektronů $\hat{T}[\rho]$, funkcionál repulze elektronů a funkcionál interakce elektronů s jádry. Jelikož externí potenciál $\nu_{ext}$ nezávisí na polohách elektronů, můžeme si posledně jmenovaný funkcionál jednoduše přepsat do tvaru, ve kterém již vystupuje elektronová hustota:
\begin{equation}
V_{ne}=\int \psi^*\nu_{ext}\psi\mathrm{d}\tau = \int \rho(\textbf{r})\nu_{ext}(\textbf{r}) \mathrm{d}\textbf{r} 
\end{equation} 


Hohenbergovy--Kohnovi teorémy dávají DFT solidní fyzikální základ, moc nás ale neposunují k praktické aplikaci, jelikož nemáme přesný funkcionál.
Praktickou cestu k DFT výpočtům ukázali až o pár let později Kohn s Shamem.

\subsection{Kohnovy--Shamovy rovnice}
První teorie, která byla založena na elektronové hustotě, spatřila světlo světa již v roce XXXX díky pánům Thomase a Fermiho. Později byla rozšířena Diracem. 

Jedním z hlavních problémů Thomase--Fermiho teorie byl nepřesný funkcionál pro kinetickou energii elektronů. Zde proto Kohn s Shamem učinila zásadní úkrok stranou.   

\subsection{Aproximace lokální hustoty}

LDA (Local density Approximation)
\begin{equation}
E_{xc}^{LDA}=\int \rho(\textbf{r})V_{xc}(\rho(\textbf{r}))\mathrm{d}\textbf{r} 
\end{equation}


\begin{equation}
V_{xc}(\rho)=V_x(\rho)+V_c(\rho)
\end{equation}
Pro výměnný člen platí pro homogenní plyn následující vztah:
\begin{equation}
V_x(\rho)=-\frac{3}{4}\left(\frac{3}{\pi}\right)^{\frac{1}{3}}\rho^{\frac{1}{3}}
\end{equation}
Pro korelační energii UEG nelze získat analytický výraz. Příslušné výpočty lze ale provést numericky a výsledek poté nafitovat. Výsledný korelační funkcionál je znám jako VWN (dle pánů Vosko-Wilk-Nusair).

LSDA....$E_{xc}=E_{xc}[\rho_\alpha\rho_\beta]$ 


\subsection{GGA a hybridní funkcionály}
GGA 
\begin{equation}
E_{xc}^{GGA}=\int \rho(\textbf{r})f(\rho,\Delta\rho\mathrm{d}\textbf{r} 
\end{equation}

Becke: B88, LYP (Lee-Yang-Parr),
kombinace: BP86, BLYP, PBE

meta-GGA

\textbf{Hybridní funkcionály}
\begin{equation}
E_X^{exact}=-\frac{1}{2}\sum_{i=1}^N\sum_{j=1}^N K_{ij}
\end{equation}

B3LYP:
\begin{equation}
E_{xc}^{B3LYP}=(1-a_0-a_x)E_x^{LDA}+a_0E_x^{exact}+a_xE_x^{B88}+(1-a_c)E_c^{VWN}+a_c E_c^{LYP}
\end{equation}
$a_0=0,2$; $a_x=0,72$; $a_c=0,81$

PBE0 (25\,\%), BMK,BHandHLYP(50\,\%)

\textbf{Long-range corrected funkcionály}
Mělo by platit, že $lim_{V_x\to \infty}=\frac{1}{r}$

\textbf{Disperzní korekce}

Grimmeho empirická korekce
\begin{equation}
E_{vdw}=-s_6\sum c^{ij}r_{ij}^{-6}
\end{equation}
Koeficient $s_6$ závisí na použitém funkcionálu, zatímco koeficienty $c_{ij}$ závisí na typu interagujících atomů.

\textbf{Dvojitě hybridní funkcionály}
S.Grimme 
\begin{equation}
E_{xc}^{hybrid}=a_1E_x^{GGA}+(1-a_1)E_x^{EXACT}+a_2E_c^{GGA}
\end{equation}
Pro tento tvar se vyřeší KS rovnice a získají KS orbitaly. Z těchto orbitalů se poté vypočítá MP2 korekce
ze vzorečku XX a přídá se k $E_{xc}^{hybrid}$

\begin{equation}
E_{xc}^{DH}=E_{xc}^{hybrid}+(1-a_2)E_c^{KS-MP2}
\end{equation}

Příkladem může být B2LYP s parametry $a_1=0,47$ a $a_2$=0,73

% Jakobův žebřík, obrázek (Genesis 28:10-12)
