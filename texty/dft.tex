Pomocí metod založených na teorii funkcionálu hustoty (dále jen DFT metod) se v posledních zhruba dvaceti let provádí většina výpočtů elektronové struktury. Popularita DFT metod je dána jejich přijatelnou výpočetní náročností, která o mnoho nepřevyšuje Hartreeho-Fockovu 
metodu. Výpočty jsou ale typicky daleko přesnější, neboť DFT metody zahrnují korelační energii. Efektivita DFT je dána tím, že nepracuje s poměrně složitou vlnovou funkcí (tj. funkcí 3N souřadnic elektronů), ale s tzv. elektronovou hustotou, což je funkce pouze tří prostorových souřadnic. 

Představme si základní pojmy. Začněme pojmem \textbf{funkcionál}. Pojďme si nejprve připomenout, jak funguje funkce. Do funkce vložíme nezávislé proměnou, tedy nějaké číslo, a na oplátku dostaneme číslo jiné, neboli závisle proměnnou. Jde tedy o zobrazení z prostoru  (kupř. reálných) čísel opět do prostoru reálných čísel. U funkcionálu je to velmi podobné, akorát na vstupu není číslo, ale funkce. Pokud to tedy řekneme více matematicky, funkcionál je zobrazení z prostoru funkcí na prostor (kupř. reálných) čísel.

Takovým jednoduchým funkcionálem je určitý integrál
$$
F[f(x)] = \int_a^b f(x) \mathrm{d}x, 
$$
Určitý integrál potřebuje dodat vstupní funkci a po jeho vyčíslení dostaneme jedno jediné číslo. Povšimněte si zde zápisu funkcionálu pomocí hranatých závorek.  
Pro složitější příklad funkcionálu nemusíme chodit daleko, stačí se podívat, jak vypadá funkcionál energie v kvantové mechanice.
$$
E[\psi] = \int \psi^*\hat{H}\psi \mathrm{d}\tau .
\label{rov:dft:efunkc}
$$
S některými funkcionály jsme se tak již setkali.

Mezi funkcemi a funkcionály existují i další podobnosti. Pojmy jako minimum a maximum funkcionálu mají prakticky stejný význam. Existuje i funkcionální analogie k dobře známé derivaci funkce --- mluvíme o \textbf{variaci funkcionálu}.
Když děláme derivaci funkce, tak se vlastně díváme, co se děje se závisle proměnnou, když trochu změníme nezávisle proměnnou.
U variace je to podobné. Zajímá nás, jak se změní hodnota funkcionálu, když mírně změníme naši funkci. Přesná definice je složitější a neuvádíme ji zde. Bude ale pro nás důležité, že pro variace platí podobné vztahy, jako pro derivace. Dá se například ukázat, že v minimu funkcionálu je jeho variace nulová. Již dobře známý variační princip (konečně víme, proč se mu tak říká!) pak můžeme napsat jednoduše pomocí variace funkcionálu energie \eqref{rov:dft:efunkc} jako
$$
\delta E[\Psi] = 0 .
$$

Nyní se můžeme vrátit k definici ústřední veličiny této kapitoly --- elektronové hustoty $\rho(\mathbf{r})$.
Elektronová hustota má na rozdíl od vlnové funkce přímý fyzikální význam. Jedná se o pravděpodobnost, že v nějakém bodě prostoru najdeme \textbf{nějaký elektron}. Je důležité si uvědomit rozdíl mezi elektronovou hustotou a čtvercem vlnové funkce, který taktéž udává hustotu pravděpodobnosti. Čtverec vlnové funkce nám udává pravděpodobnost, že první elektron má spin $m_{s1}$ a je v bodě $\mathbf{r}_1$, druhý elektron má spin $m_{s2}$ a je v bodě $\mathbf{r}_2$ atd. Jedná se tedy o mnohem složitější veličinu, která závisí na celkem $4N$ proměnných, zatímco elektronová hustota závisí jen na třech proměnných.

Elektronová hustota souvisí s vlnovou funkcí systému dle vztahu
\begin{equation}
\rho=N \int |\psi(\textbf{r}_1,\textbf{r}_2,...,\textbf{r}_n)|^2 \mathrm{d}\textbf{r}_2\dots\mathrm{d}\textbf{r}_n .
\label{rov:dft:defrho}
\end{equation}
Integrujeme tedy čtverec vlnové funkce přes všechny elektrony kromě prvního. Jak již bylo řečeno, chceme pravděpodobnost najití jakéhokoli elektronu, poloha ostatních nás nezajímá, a musíme tedy přes ně prointegrovat. První elektron ale není nijak odlišný od ostatních, stejně dobře bychom ale mohli udělat to samé pro druhý elektron. Jelikož je vlnová funkce antisymetrická, došli bychom ke stejnému výsledku. Z toho vyplývá násobící faktor $N$ před integrálem.
Z této definice hned plyne několik důležitých vlastností.

\begin{itemize}
\item Elektronová hustota je nezáporná veličina, platí tedy
\begin{equation}
\rho(\mathbf{r})  > 0 .
\end{equation}
\item Pokud zintegrujeme elektronovou hustotu přes celý prostor, dostaneme počet elektronů v systému jako
\begin{equation}
\int \rho\mathrm{d}r = N
\end{equation}

\item V poloze jader má elektronová hustota maxima.
\item Pro tato maxima platí
%TOPETR: v poznamkach mas, ze to je nejaky teorem, ale neprectl jsem jaky....
\begin{equation}
\lim_{r_i \to 0} \left[ \frac{\delta}{\delta r}+2Z_A\right]\hat{\rho(r)}=0, 
\end{equation}
kde $\hat{\rho}$ je angulárně zprůměrovaná hodnota elektronové hustoty a $Z_A$ je náboj příslušného atomového jádra.
\end{itemize}

Z těchto vlastností vidíme, že pokud známe elektronovou hustotu, tak zároveň také můžeme zjistit počet elektronů, polohu jader i jejich náboj. To jsou ale přesně ty informace, které jsou potřeba ke specifikaci molekulárního hamiltoniánu
\begin{equation}
\hat{H}=\sum_{i=1}^N -\frac{1}{2}\Delta_i+\sum_{i=1}^N\sum_{j=i+1}^N\frac{1}{r_{ij}}-\sum_{i=1}^N\sum_{A=1}^K \frac{Z_A}{r_{iA}} ,
\label{rov:dft:molham}
\end{equation}
kde jsme vynechali pro jednoduchost člen popisující odpuzování jader, jenž je stejně v rámci Bornovy--Oppenheimerovy aproximace roven konstantě. Pokud ale elektronová hustota jednoznačně určuje hamiltonián, tak poté z řešení SCHR taky energii a vlnovou funkci, a tudíž všechny potřebné veličiny. Podobným způsobem se zřejmě ubíraly úvahy Hohenberga a Kohna, kteří postavili teorii DFT na pevné fyzikální základy.

\subsection{Hohenbergovy--Kohnovy teorémy}

V roce 1964 publikovali Hohenberg s Kohnem slavný článek, který odstartoval vývoj DFT metod.
V tomto článku byly dokázány dva důležité teorémy. K jejich lepšímu pochopení si ještě musíme definovat pojem \textbf{externího potenciálu} $\nu_{ext}$. Ten má pro případ hamiltoniánu \eqref{rov:dft:molham} tvar
\begin{equation}
\nu_{ext} = \sum_{i=1}^N\sum_{A=1}^K \frac{Z_A}{r_{iA}} .
\end{equation} 
Jedná se tedy o celkovou interakci elektronů s coulombickým potenciálem atomových jader. Jelikož se v DFT na vše díváme z pohledu elektronů, tak se tomu potenciálu říká externí, jelikož nepochází ze samotných elektronů. Je důležité si uvědomit, že externí potenciál vlastně definuje hamiltonián \eqref{rov:dft:molham}, neboť ostatní členy triviálně závisejí pouze na počtu elektronů v systému. Nyní již můžeme přejít ke slíbeným teorémům.

\textbf{První Hohenbergův--Kohnův teorém} hovoří o významnosti elektronové hustoty. Zní takto:

\bigskip
\noindent \uv{\textbf{Pro libovolný systém interagujících elektronů je externí potenciál $\nu_{ext}$ jednoznačně určen elektronovou hustotou} (až na konstantu)}

\bigskip
Důsledek tohoto tvrzení jsme si již naznačili dříve. Pokud máme jednoznačně daný externí potenciál, pak také známe hamiltonián a můžeme v principu spočítat vlnovou funkci i energii, které jsou tudíž jednoznačně určeny pouze elektronovou hustotou. \textbf{Elektronová energie je tedy jednoznačným funkcionálem elektronové hustoty} neboli v matematickém zápisu
\begin{equation}
E_{el}=E_{el}[\rho] .
\end{equation}
Tvar tohoto funkcionálu je nezávislý na externím potenciálu a je tedy univerzální pro všechny atomy a molekuly.
\textbf{ToPS:Z čeho plyne ta univerzalita?}
Pokud bychom tedy tento funkcionál znali a znali také správnou hustotu, tak bychom mohli získat energii i bez řešení Sch\"{o}dingerovy rovnice a hledání vlnových funkcí.

\bigskip
Pro důkaz tohoto teorému si musíme ještě odvodit pomocný vztah pro integraci externího potenciálu vynásobeného vlnovou funkcí.
Rozepišme si nejprve externí potenciál na součet jednoelektronových příspěvků:
\begin{equation}
\nu_{ext}=\sum_{i=1}^N \nu_i = \sum_{i=1}^N \sum_{A=1}^K \frac{Z_A}{r_{iA}}  
\label{rov:dft:nui}
\end{equation}
Pojďme si nyní vyčíslit integrál
\begin{equation}
\int \nu_{ext} \psi(\textbf{r}_1,...,\textbf{r}_N)^2\mathrm{d}\tau .
\end{equation}
Dosazením vztahu \eqref{rov:dft:nui} a vhodným přeuspořádáním mnohonásobného integrálu dostaneme
\begin{equation}
\sum_i^N \int \nu_i \left[\int \psi(\mathbf{r}_1,...,\mathbf{r}_N)\prod_{j\neq i}\mathrm{d}\textbf{r}_j\right] \mathrm{d}\textbf{r}_i .
\end{equation}
Výraz v hranaté závorce není ale nic jiného než elektronová hustota $\rho$ (až na násobný faktor $N$), dostáváme tedy finální vztah
\begin{equation}
\int \nu_{ext} \psi(\textbf{r}_1,...,\textbf{r}_N)^2\mathrm{d}\tau = \sum_{i=1}^N \int \nu_i(\textbf{r}_1)\frac{\rho(\textbf{r}_i)}{N}= \int \nu_i(\textbf{r})\rho(\textbf{r}) \mathrm{d}r .
\label{rov:dft:intnupsi}
\end{equation}

\bigskip 
\textbf{Důkaz HK1:} Důkaz se provádí sporem. Předpokládejme, že jedné dané hustotě $\rho$ přísluší dva různé externí potenciály $\nu_{ext1}$ a $\nu_{ext2}$, kterým přísluší hamiltoniány $\hat{H}_1$ a $\hat{H}_2$ a vlnové funkce $\psi_1$ a $\psi_2$.  Pak díky variačnímu principu musí platit

\begin{equation}
E_1 = \int \psi_1^* \hat{H}_1 \psi_1 \mathrm{d}\tau < \int \psi_2^* \hat{H}_1 \psi_2 . \mathrm{d}\tau ,
\end{equation}
protože když k hamiltoniánu $\hat{H_1}$ přiřadíme funkci $\psi_2$, která není jeho vlastní funkcí základního stavu, tak musíme dostat větší energii. Nyní chytře přepíšeme pravou stranu nerovnosti jako  
\begin{equation}
\int \psi_2^* \hat{H}_1 \psi_2 = \int \psi_2^* \hat{H}_2 \psi_2\mathrm{d}\tau  + \int \psi_2^* \left[\hat{H}_1-\hat{H}_2\right] \psi_2\mathrm{d}\tau = E_2 + \int \rho(\mathbf{r})(\nu_1-\nu_2)\mathrm{d}\mathbf{r}  
\end{equation}
V poslední rovnosti jsme využili toho, že oba hamiltoniány se liší pouze externím potenciálem, a použili jsme vztah \eqref{rov:dft:intnupsi}.

\noindent Vyšlo nám tedy, že platí nerovnost
\begin{equation}
E_1 < E_2+\int \rho(\mathbf{r})(\nu_1-\nu_2)\mathrm{d}\mathbf{r}
\label{rov:HK1_1}
\end{equation}
Tu samou argumentaci bychom ale mohli také aplikovat opačně a dostali bychom
\begin{equation}
E_2 < E_1+\int \rho(\mathbf{r})(\nu_2-\nu_1)\mathrm{d}\mathbf{r} .
\label{rov:HK1_2}
\end{equation}
Sečtením obou rovnic \eqref{rov:HK1_1} a \eqref{rov:HK1_2} tedy dostáváme
\begin{equation}
E_1 + E_2 < E_2 + E_1 ,
\end{equation}
což nemůže platit, a tedy náš původní předpoklad o existenci dvou různých externích potenciálů je také chybný.
\hfill {\footnotesize $\blacksquare$}

Formálně můžeme funkcionál energie napsat jako
\begin{equation}
E=\int \psi^*\hat{H}\psi \mathrm{d}\tau = \int \psi^*\hat{T}\psi\mathrm{d}\tau + \int \psi^*\hat{V_{el}}\psi\mathrm{d}\tau + \int \psi^*\nu_{ext}\psi\mathrm{d}\tau=\hat{T}[\rho]+\hat{V}_{ee}[\rho]+\hat{V}_{ne}[\rho] .
\end{equation}
Vidíme, že energie se stejně jako příslušný hamiltonián \eqref{rov:dft:molham} skládá ze tří různých příspěvků. Funkcionál energie tedy můžeme rozdělit na funkcionál kinetické energie elektronů $\hat{T}[\rho]$, funkcionál repulze elektronů $\hat{V}_{ee}[\rho]$ a funkcionál interakce elektronů s jádry (obecně s externím potenciálem) $\hat{V}_{ne}[\rho]$. 
Explicitní tvar posledně jmenovaného funkcionálu jsme si již vlastně odvodili rovnicí \eqref{rov:dft:intnupsi} a platí tedy
\begin{equation}
\hat{V}_{ne}[\rho] = \int \nu_i(\textbf{r})\rho(\textbf{r}) \mathrm{d}r .
\end{equation}
Součet zbylých dvou funkcionál nazýváme Hohenbergovým--Kohnovým funkcionálem a značíme $F_{HK}[\rho]$. 
Celkový funkcionál energie tedy můžeme zapsat jako
\begin{equation}
E[\rho] = \hat{T}[\rho]+\hat{V}_{ee}[\rho]+\hat{V}_{ne}[\rho] = F_{HK}[\rho] + \int \nu_i(\textbf{r})\rho(\textbf{r}) \mathrm{d}r .
\end{equation}

Přesný tvar Hohenbergova--Kohnova funkcionálu $F_{HK}[\rho]$ bohužel není dodnes znám. 
Ovšem i kdybychom jej znali, tak nám stále něco schází k jeho úspěšného použití. Nevíme totiž, jak získat pro daný systém správnou elektronovou hustotu $\rho$, kterou bychom do něj mohli dosadit. Samozřejmě kdybychom znali vlnovou funkci, tak stačí dosadit do definičního vztahu \eqref{rov:dft:defrho}. Tomu se ale právě chceme vyhnout! Smyslem celé teorie funkcionálu hustoty je vyhnout se přímému řešení elektronové SCHR.

Cestu k elektronové hustotě nám dává druhý teorém od Hohenberga a Kohna, který zní takto:

\bigskip
\textbf{Předpokládejme, že danému externímu potenciálu $\nu_{ext}$ přísluší elektronová hustota $\rho_0$. Pak pro jakoukoli jinou elektronovou hustotu\footnote{Přesněji řečeno, daná elektronová hustota musí být takzvaně $\nu$--{reprezentovatelná}, neboli musí k ní příslušet nějaký externí potenciál. Pro funkce, které toto nesplňují, 2. HK teorém neplatí.} $\rho^{\prime}$ bude platit:}
\begin{equation}
E[\rho_0] < E[\rho^{\prime}]
\label{rov:dft:HK2}
\end{equation}

Nejedná se o nic jiného než variantu variačního principu, který taky využijeme k důkazu.

\bigskip
\textbf{Důkaz:} Z prvního HK teorému plyne, že jakákoli funkce $\rho^{\prime}$ patří k externí potenciálu $\nu_{ext}^{prime}$, který je odlišný od $v_{ext}$), a přísluší k němu vlnová funkci $\psi^{prime}$. Pokud ale pro tuto vlnovou funkci vyčíslíme energii, pak nám z již známého variačního principu plyne
\begin{equation}
\int \psi^{\prime *} \hat{H} \psi^{\prime} \mathrm{d}\tau = E[\rho] > E [\rho_0],
\end{equation}
což jsme chtěli dokázat. \hfill {\footnotesize $\blacksquare$}

Druhý HK teorém nám tedy dává principiální návod, jak hledat elektronovou hustotu. Budeme hledat přes všechny možné hustoty a správná bude ta, která nám dá nejnižší energii.

Hohenbergovy--Kohnovy teorémy dávají DFT solidní fyzikální základ, moc nás ale neposunují k praktické aplikaci, jelikož neznáme přesný funkcionál $F_{HK}[\rho]$.
Praktickou cestu k DFT výpočtům ukázali až o pár let později Kohn s Shamem.

\subsection{Kohnovy--Shamovy rovnice}

Nyní stojíme před zásadním problémem najití alespoň přibližného funkcionálu $F_{HK}[\rho]$, ve kterém je zahrnuta kinetická energie elektronů, klasická Coulombická interakce mezi elektrony a dále korelační a výměnné efekty. 
%Obecně bychom chtěli spočítat co největší část energie pomocí známých dobře definovaných přibližných vztahů a zbylé části poté můžeme modelovat a odchylky poté můžeme modelovat třeba semiempiricky.(\footnote{Tento přístup by se dal přirovnat k situaci v chemické termodynamice, ve které počítáme se vzorečky platnými pro ideální chování a odchylky schováváme do aktivitních koeficientů).
Ukázalo se, že největší potíže činí dostatečně přesné vyjádření funkcionálu pro kinetickou energii.
Tento problém je tak zásadní, že budeme muset částečně obětovat náš původní cíl a vrátit se k molekulovým orbitalům.
Pokud totiž máme molekulové orbitaly, tak pomocí nich můžeme vyčíslit kinetickou energii dle vztahu
\begin{equation}
E_{kin}=\sum_{i=1}^N \int \varphi\frac{1}{2} \Delta_i \varphi \mathrm{d}\textbf{r}_i .
\end{equation}
S takto postavenou teorií přišli v roce 1965 Kohn s Shamem.

Obecná strategie odvození Kohnovy--Shamovy proceduru je následující. Definuje se fiktivní systém neinteragujících elektronů (podobně jako v Hartreeho--Fockově teorii zde elektrony interagují pouze skrze efektivní potenciál), který je zvolen tak, aby jeho elektronová hustota bylo rovna elektronové hustotě reálného systému. 
Funkcionál energie se rozepíše následujícím způsobem:
\begin{equation}
E[\rho]= T_{n}[\rho] + \frac{1}{2}\int \int \frac{\rho(\textbf{r}_1)\rho(\textbf{r}_2)}{r_{12}} + V_{nekl}(\rho) +\left[T[\rho]-T_{n}[\rho]\right] + \int \nu \rho \mathrm{d}\textbf{r},
\label{rov:dft:KSfunkc}
\end{equation}
kde $T_{n}[\rho]$ kinetická energie neinteragujícího systému, druhý člen odpovídá klasické mezi-elektronové repulzi (násobí se jednou polovinou, aby se interakce nezapočítávaly dvakrát), $V_{nekl}$ zahrnuje korelační a výměnnou energii elektronů, člen v hranaté závorce je rozdíl kinetických energií reálného a neinteragujícího systému  a poslední člen odpovídá interakci elektronů s jádry. Pokud všechny neznámé členy dáme dohromady dostaneme tzv. korelačně--výměnný potenciál
\begin{equation}	
E_{XC}[\rho]=\left[T[\rho]-T_{n}[\rho]\right] +  V_{nekl} .
\label{rov:dft:exc}
\end{equation}
% V_{ee}(\rho) - \frac{1}{2}\int \int \frac{\rho(\textbf{r}_1)\rho}{\textbf{r_{12}}}
Pokud na funkcionál \eqref{rov:dft:KSfunkc} nyní aplikujeme variační princip, tak dostaneme rovnice, které mají stejný tvar jako rovnice
pro systém neinteragujících elektronů. Jenže pro tento systém známe řešení! Stačí vyřešit vyřešit příslušnou jedno--elektronovou  	 Schr\"{o}dingerovu rovnici, velmi podobnou rovnicím Hartreeho--Focka
\begin{equation}
\left(-\frac{1}{2}\Delta_i + V_{eff} \right) \varphi_i =\epsilon_i \varphi_i ,
\label{rov:dft:KSeq}
\end{equation}
kde $V_{eff}$ je efektivní potenciál, pro který platí
\begin{equation}
V_{eff}=\nu_{ext}+\frac{1}{2}\frac{\rho(\textbf{r}^{\prime})}{|\textbf{r}-\textbf{r}^{\prime}|}\mathrm{d}\textbf{r}^{\prime}+u_{xc} ,
\end{equation}
kde $u_{xc}$ je funkcionální derivace (variace) výměnně--korelačního funkcionálu \ref{rov:dft:exc} tvar tohoto potenciálu byl odvozen tak, aby byly elektronové hustoty reálného i fiktivního systému stejné.
Rovnice \ref{rov:dft:KSeq} se nazývají Kohnovy--Shamovy a řeší se podobně jako rovnice Hartreeho--Fockovy rozvojem do báze AO.
Získáme tak sadu molekulových orbitalů, ze kterých pak dostaneme elektronovou hustotu dle vztahu (uvádíme bez důkazu)
\begin{equation}
\rho(\textbf{r}) = \sum_{i=1}^N |\varphi|_i^2
\label{rov:dft:KSrho}
\end{equation}
%Jelikož neinteragující systém byl definován tak, aby 
Tuto hustotu pak můžeme dosadit do funkcionálu \eqref{rov:dft:KSfunkc} a získáme tak požadovanou energii.

Kohn--Shamův přístup je v principu přesný, pokud bychom znali přesný tvar výměnně--korelačního funkcionálu $E_{XC}[\rho]$.
Ten sice neznáme, ale existuje spousta vztahů přibližných, o kterých pojednávají další kapitoly.


\subsection{Aproximace lokální hustoty}

LDA (Local density Approximation)
\begin{equation}
E_{xc}^{LDA}=\int \rho(\textbf{r})V_{xc}(\rho(\textbf{r}))\mathrm{d}\textbf{r} 
\end{equation}


\begin{equation}
V_{xc}(\rho)=V_x(\rho)+V_c(\rho)
\end{equation}
Pro výměnný člen platí pro homogenní plyn následující vztah:
\begin{equation}
V_x(\rho)=-\frac{3}{4}\left(\frac{3}{\pi}\right)^{\frac{1}{3}}\rho^{\frac{1}{3}}
\end{equation}
Pro korelační energii UEG nelze získat analytický výraz. Příslušné výpočty lze ale provést numericky a výsledek poté nafitovat. Výsledný korelační funkcionál je znám jako VWN (dle pánů Vosko-Wilk-Nusair).

LSDA....$E_{xc}=E_{xc}[\rho_\alpha\rho_\beta]$ 


\subsection{GGA a hybridní funkcionály}
GGA 
\begin{equation}
E_{xc}^{GGA}=\int \rho(\textbf{r})f(\rho,\Delta\rho\mathrm{d}\textbf{r} 
\end{equation}

Becke: B88, LYP (Lee-Yang-Parr),
kombinace: BP86, BLYP, PBE

meta-GGA

\textbf{Hybridní funkcionály}
\begin{equation}
E_X^{exact}=-\frac{1}{2}\sum_{i=1}^N\sum_{j=1}^N K_{ij}
\end{equation}

B3LYP:
\begin{equation}
E_{xc}^{B3LYP}=(1-a_0-a_x)E_x^{LDA}+a_0E_x^{exact}+a_xE_x^{B88}+(1-a_c)E_c^{VWN}+a_c E_c^{LYP}
\end{equation}
$a_0=0,2$; $a_x=0,72$; $a_c=0,81$

PBE0 (25\,\%), BMK,BHandHLYP(50\,\%)

\textbf{Long-range corrected funkcionály}
Mělo by platit, že $lim_{V_x\to \infty}=\frac{1}{r}$

\textbf{Disperzní korekce}

Grimmeho empirická korekce
\begin{equation}
E_{vdw}=-s_6\sum c^{ij}r_{ij}^{-6}
\end{equation}
Koeficient $s_6$ závisí na použitém funkcionálu, zatímco koeficienty $c_{ij}$ závisí na typu interagujících atomů.

\textbf{Dvojitě hybridní funkcionály}
S.Grimme 
\begin{equation}
E_{xc}^{hybrid}=a_1E_x^{GGA}+(1-a_1)E_x^{EXACT}+a_2E_c^{GGA}
\end{equation}
Pro tento tvar se vyřeší KS rovnice a získají KS orbitaly. Z těchto orbitalů se poté vypočítá MP2 korekce
ze vzorečku XX a přídá se k $E_{xc}^{hybrid}$

\begin{equation}
E_{xc}^{DH}=E_{xc}^{hybrid}+(1-a_2)E_c^{KS-MP2}
\end{equation}

Příkladem může být B2LYP s parametry $a_1=0,47$ a $a_2$=0,73

\textbf{Jakobův žebřík, obrázek (Genesis 28:10-12)}
