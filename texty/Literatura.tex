\subsection*{Co číst dále?}
Předložený text není ucelenou učebnicí kvantové teorie molekul, nýbrž pouhými sebranými poznámkami k~přednáškám. Jednotlivé kapitoly a obsah docela dobře korespondují s~látkou probíranou na přednáškách, jde ale pořád toliko o~fragmenty žádající si rozšiřující informaci. I~přes veškerou snahu může být navíc výklad veden způsobem pro čtenáře málo srozumitelným. Čtenáři proto naléhavě doporučujeme souběžné studium y dalších pramenů. Níže podáváme stručný komentovaný přehled dostupné literatury.

   	
\subsubsection*{Základy kvantové teorie v~chemii}

Studium kvantové chemie, tedy aplikace kvantové teorie na chemické otázky, vyžaduje alespoň rámcovou znalost kvantové teorie jako takové. V~této části proto zmíníme některé z~publikací, které čtenáři pomohou se v~základních principech kvantové teorie zorientovat.

\begin{itemize}

\item David O. Hayward, \textit{Quantum Mechanics for Chemists} RSC, 2002. Tento rozsahem nevelký text lze doporučit jako první text ke studiu kvantové teorie. Je psát čtivým, srozumitelným jazykem a studentovi-začátečníkovi bude výtečným pomocníkem.   
\item Peter Atkins, Julio de Paula, \textit{Fyzikální chemie} VŠCHT Praha, 2013. Kvantová chemie je běžnou součástí základních (tj. bakalářských) kurzů fyzikální chemie. Proto nás nepřekvapí, že asi třetina této nejznámější učebnice fyzikální chemie je věnována právě kvantové chemii. Významná část našeho kurzu je právě v~Atkinsově učebnici pokryta. Učebnice je zároveň pro začínající studenty a typicky je tak dobře srozumitelná.
\item Thomas Engel, \textit{Quantum Chemistry and Spectroscopy} Prentice Hall, 2010. Engelův text je součástí učebnice fyzikální chemie, vyšel ovšem i v~samostatném svazku. Jde opět o~úvodní kurz. Část věnovaná kvantové teorii se autorovi mimořádně povedla, kniha je velmi pěkně zpracována i po grafické stránce. 
\item
\item
\item
\item
\item


\end{itemize}


\subsubsection*{Doplňující a rozšiřující literatura}
Mnohé rozšiřující informace nalezne čtenář v~učebnicích Atkinse a Engela, o~kterých byla řeč výše. V~této části okomentujeme pokročilejší texty zaměřené na kvantovou chemii.
 
\begin{itemize}

\item Attila Szabo, Neil S. Ostlund,\textit{Modern Quantum Chemistry} Dover, 1996. Dnes již klasické dílo, které v~sevřené podobě popisuje pokročilé metody kvantové chemie. Kniha se dobře čte a doverská edice je také finančně dosažitelná. Z~dnešního pohledu možná zarazí nepřítomnost teorie funkcionálu hustoty. 
\item Ira N. Levine, \textit{Quantum Chemistry} Pearson/Prentice Hall, 2009. Poctivá učebnice kvantové chemie s~řadou příkladů. Rozsahem velmi dobře odpovídá přednáškám z~kvantové chemie. 
\item Peter W. Atkins, Ronald S. Friedman, \textit{Molecular Quantum Mechanics} Oxford University Press, 2010. Tato učebnice navazuje na Atkinsův základní kurz fyzikální chemie. Kromě kvantové chemie se čtenář seznámí i s~jinými aspekty molekulární kvantové teorie, kupříkladu s~teoretickou spektroskopií.  
\item John P. Lowe, Kirk A. Peterson, \textit{Quantum Chemistry} Elsevier, 2006. Zajímavá učebnice, vhodná zejména pro ty ze čtenářů, kteří si rádi probírané koncepty sami vyzkouší. Autor se hojně věnuje H\"uckelově teorii, se kterou dokáže překvapivě mnoho.
\item Jean-Pierre Launay, Michel Verdaguer, \textit{Electrons in Molecules. From Basic Principle to Molecular Electronics} Oxford University Press, 2014. Kvantová chemie v~moderním kontextu vědy o~materiálech. 
\item Rudolf Polák a Rudolf Zahradník, \textit{Kvantová chemie. Základy teorie a aplikace} SNTL, 1985. Klasické česky psané dílo, které se i po letech dobře čte, mimo jiné i díky svižnému slohu.
\item Jiří Fišer, \textit{Úvod do kvantové teorie}. Academia, 1983. Kniha vhodná zejména pro ty ze čtenářů, kteří by se rádi seznámili s~metodami lineární algebry v~kvantové chemii.
\item Lubomír Skála, \textit{Kvantová teorie molekul} Univerzita Karlova, 1995. Rozsahem nevelká skripta obsahují detailní a srozumitelný popis základních kvantově-chemických přístupů.

\end{itemize}
