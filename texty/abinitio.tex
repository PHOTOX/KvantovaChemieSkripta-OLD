Doteď jsme se zabývali převáže základy kvantové teorie 
Většinu toho jsme si byli schopni odvodit a vypočítat analyticky. 
Pokud ale chceme kvantovou teorii aplikovat na něco více než atomy a jednoduché molekuly,
neobejdeme se bez počítače, který za nás bude počítat složité rovnice.

Naprosto základní úlohou kvantového chemika je řešení elektronové Schr\"{o}dingerovy rovnice XX pro danou molekulu. Existují v zásadě tři přístupy k tomuto problému. Buď se snažíme tuto rovnici řešit přímo pomocí chytrých aproximací (např. pomocí HF metody) a bez dodatečných umělých parametrů. Tyto metody se nazývají \textit{ab initio} (latinsky \uv{od počátku}, jelikož k jejich aplikaci nám stačí znát pouze základní fyzikální konstanty jako je hmotnost a náboj elektronu nebo Planckova konstanta. Další třidou jsou semiempirické metody, které jsou již více aproximativní a obsahují parametry, jejichž hodnoty se získávají buď z experimentálních dat nebo z přesnějších \textit{ab initio} výpočtů. Poslední přístupem jsou metody založené na takzvané teorii funkcionálu hustoty (Density Functional Theory --- DFT). Tento přístup je fundamentálně odlišný od předchozích dvou, jelikož není založen na hledání vlnové funkce. Místo toho se soustředí na mnohem jednodušší veličinu -- elektronovou hustotu. DFT metoda je často řazena mezi \textit{ab initio} metody, jelikož v principu vede k přesnému řešení a praktická přesnost je taktéž na úrovni \textit{ab initio} metod. Na druhou stranu ale většina DFT metod obsahuje několik empirických parametrů, které se často fitují na experimentální data. Proto si DFT probereme ve zvláštní kapitole.  

V této kapitole se zaměříme na \textit{ab initio} metody.
Nejjednodušší \textit{ab initio} metodou je metoda Hartreeho--Fockova, která již byla představena v kapitolách X a XX. V praxi se dnes používá spíše méně, ale stále slouží jako odrazový můstek pro většinu ostatních přesnějších\textit{ab initio} metod, tudíž si ji zde zopakujeme a probereme hlouběji.

Nejprve je ale důležité ujasnit si notaci, kterou budeme v následujících kapitolách používat, abychom se ve všech těch řeckých písmenech neztratili. Celkovou vlnovou funkci budeme značit symbolem $\phi$.  zatímco molekulové orbitaly (MO) budeme značit $\varphi$. Každý MO je součinem prostorové části $\phi$ a spinové části $\alpha$ nebo $\beta$. V rámci aproximace MO-LCAO poté každou prostorovou část MO rozvíjíme do báze atomových orbitalů $\chi_i$. Abychom si ušetřili práci s psaním rovnic, budeme odteď využívat atomové jednotky.

\begin{table}[ht]
\centering
\caption{Základní symboly použité následujících kapitolách}
\begin{tabular}{|c|l|c|}
\hline 
\rule[-1ex]{0pt}{2.5ex} Symbol & 	Význam	& Vlastnosti \\ 
\hline 
\rule[-1ex]{0pt}{2.5ex} $\psi$ & Celková el. vln. funkce  & $\psi$ = $|\phi_1 \phi_2 \cdots \phi_N |$ \\ 
\hline 
\rule[-1ex]{0pt}{2.5ex} $\phi$ & Molekulový orbital & 4$\varphi=\phi \alpha $\\ 
\hline 
\rule[-1ex]{0pt}{2.5ex} $\phi$ & Prostorová část MO & $\phi=\sum_{i=1}^N \chi $ \\ 
\hline 
\rule[-1ex]{0pt}{2.5ex} $\alpha$, $\beta$  & Spin--orbitaly & \\ 
\hline 
\rule[-1ex]{0pt}{2.5ex} $\chi$ & atomové orbitaly & $\int |\varphi|^2 \mathrm{d}\textbf{r} = 1 $
\hline
\label{tab:vlnfunkce}
\end{tabular} 
\end{table}

\subsection{Hartreeho--Fockova metoda}

Zopakujme si nejprve základní předpoklady a odvození HF teorie. Začneme tím, že celkovou vlnovou funkci předpokládáme ve formě Slaterova determinantu (SD)
\begin{equation}
\psi^HF=\frac{1}{\sqrt{N!}}\begin{vmatrix}
\varphi_1(1) & \varphi_1(2) & \cdots & \varphi_1(N) \\
\varphi_2(1) & \varphi_2(2) & \cdots & \varphi_2(N) \\
\vdots & \vdots & \vdots & \vdots \\
\varphi_N(1) & \varphi_N (2) & \cdots & \varphi_N(N)
\end{vmatrix}.
\end{equation}
Jelikož SD má $N!$ členů, kde $N$ je počet elektronů, je třeba použít normalizační faktor $\frac{1}{\sqrt{N!}}$.

Na tento tvar vlnové funkce chceme nyní aplikovat variační princip. Po dosazení do variačního vzahu pro energii by nám po určité námaze vyšel vztah
\begin{equation}
E^{HF}=  XXX 
\end{equation}

Tento vztah byl odvozen za předpokladu, že jsou molekulové orbitaly ortogonální. Tento požadavek je ryze praktického rázu, ke stejnému výsledku bychom došli i bez něj, ale mnohem obtížněji. Aplikací variačního počtu na předchozí rovnici (více o funkcionálech a variacích si povíme v kapitole o DFT) dostaneme Fockovy rovnice: \textbf{nejsem si jist, zda zde jsou spin orbitaly nebo ne, musim zkontrolovat}
\begin{equation}
\hat{F}\phi_i(1) = \epsilon_i \phi_i   
\end{equation}

kde Fockův operátor $\hat{F}$ vypadá následovně:

\begin{eqnarray}
\hat{F}_i = \hat{h}_i+\sum_{j\neq i} \hat{J}_j - \delta(s_i,s_j) \hat{K}_j \\
\hat{h}_i = -\frac{1}{2}\Delta_i - \sum_{A}\frac{Z_A}{r_{iA}}
\hat{J}_j=\int \frac{|\varphi |^2}{r_{ij}}\mathrm{d}\textbf{r}_j \\
\hat{K}_{ij} = \int \frac{\phi_j^*(\mathbf{r}_j)\phi_i(\mathbf{r}_i)}{r_{ij}}\mathrm{d}\textbf{r}_j
\end{eqnarray}

Fockovy rovnice jsou složité integro-diferencíální rovnice pro neznámé funkce $\phi_i$

Jelikož molekuly jsou složeny z atomů, vhodnou bází jsou atomové orbitaly.
Dnes se již téměř výhradně používají Gaussovy funkce, o kterých již byla řeč v kapitole XX.


Jak je to s přesností HF metody?

Rozdíl přesného řešení nerelativistické Schr\"{o}dingerovy rovnice a HF metody se nazývá korelační energie. Je důsledkem toho, že jsme zanedbali okamžitou mezi--elektronovou repulzi. Pojďme se na to podívat podrobněji. V reálné molekule bychom neočekávali, že se dva elektrony budou nacházet na stejném místě, takovéto uspořádání by totiž dle Coulombova zákona mělo nekonečnou energii. 


Na konec je třeba podotknout, že celý koncept molekulových orbitalů, na které jsou chemici již tak zvyklí, je založen na HF teorii, a je tudíž pouze aproximací. Jakmile se budeme snažit dosáhnout přesnějších výsledků, začne se nám úhledný obrázek elektronů sedících v molekulových orbitalech rozpadat.

\subsection{Metoda kofigurační interakce}

\textbf{nevím, jak přeložit pojem size-consistency}

Jednou z možností, jak se dostat ke korelační energii je vyjít z HF metody a opět využít variační princip. Jakou ale použít bázi? Dá se ukázat, že množina všech Slaterových determinantů, které můžeme vytvořit z HF molekulových orbitalů, je úplnou bází daného Hilbertova prostoru, a můžeme tedy do ní expadovat naši hledanou vlnovou funkci, která bude mít tvar
\begin{equation}
\psi_{CI}=c_0\psi_0+\sum_a\sum_m c_a^m\psi_a^m+\sum c_{ab}^{mn}\psi_{ab}^{mn}+\dots
\label{rov:CIrozvoj}
\end{equation}
První člen je HF řešení a další členy odpovídající determinantům, kde jsme excitovali jeden nebo více elektronů z obsazeného do neobsazeného orbitalu (viz obrázek XX). Druhý člen odpovídá mono--excitacím (obr. xx.A), druhý člen odpovídá bi--excitacím (obrázek XX.B) atp. Jednotlivým SD můžeme také říkat konfigurace, odtud název metody konfigurační interakce.

\bigskip

\textbf{Obrazek excitovaných determinantů}

\bigskip

Jak nyní určíme koeficienty $c$? Stačí nám k tomu variační princip. Jelikož vlnová funkce na těchto koeficientech závisí lineárně, aplikací variačního principu dostaneme již dříve zmíněné sekulární rovnice.
\begin{eqnarray}
\mathbb{H}\mathbf{c}=E\mathbf{c} \nonumber \\
\sum_j (H_ij-E_i\delta_{ij})c_i=0 \quad \forall i
\end{eqnarray}
Výsledné matice jsou obrovské (milion a více členů), ale existují efektivní metody výpočetní lineární algebry, které si s nimi dokáží poradit.

Pokud bychom použili nekonečnou bázi AO a zahrnuli všechny možné Slaterovy determinanty (kterých by bylo taktéž nekonečné mnoho), dostali bychom přesné řešení Schr\"{o}dingerovy rovnice. V praxi ale musíme použít bázi konečnou, máme tedy i konečný počet molekulových orbitalů. Pokud v rámci této báze budeme uvažovat všechny možné SD, mluvíme o metodě plné konfigurační interakce (\textit{Full Configuration Interaction} -- FCI). Zvětšováním báze se metoda FCI může libovolně přiblížit přesnému řešení. Metoda FCI je ale extrémně výpočetně náročná (škáluje se exponenciálně s počtem orbitalů) a lze ji tedy použít pouze pro malé molekuly s malou bází.

Pro praktické výpočty tedy musíme počet excitací omezit. Pokud se omezíme pouze jednonásobné a dvojnásobné excitace, dostaneme metodu CISD (\textit{Configuration Interaction Singles}\& \textit{Doubles}), která byla v minulosti hojně využívána.

Dnes se metody konfigurační interakce užívají méně, jelikož mají některé nevýhody oproti jiným metodám. 
\begin{itemize}
\item Metoda FCI je sice přesná, ale v praxi díky exponenciálnímu škálování nepoužitelná.
\item Rozvoj \ref{rov:CIrozvoj} konverguje velmi pomalu. Pro dobrou přesnost bychom chtěli minimálně kvadruple--excitace.
\item Pokud rozvoj \ref{rov:CIrozvoj} někde utneme, tak výsledná metoda není size--konzistentní. Co tento pojem znamená? Zjednodušeně řečeno, pokud metoda není size--konzistentní, tak její přesnost bude záviset na velikosti systému. Pro malou molekulu můžeme například metodou CISD zachytit přes 90 \% korelační energie, ale pro větší molekulu mnohem méně. Pokud je metoda size--konzistentní, tak by také mělo platit, že pokud ji aplikujeme na dvě nekonečně vzdálené molekuly (není tedy mezi nimi žádná interakce), tak bychom měli dostat přesně dvojnásobek energie jedné molekuly. Na příkladu dvou atomů helia lze ale snadno ukázat, že toto pro metodu CISD není splněno. Pro jeden atom He totiž metoda CISD odpovídá metodě FCI, jelikož máme pouze dva elektrony, které můžeme excitovat. Pro dva nekonečně vzdálené atomy helia již ale toto neplatí, jelikož máme už čtyři elektrony a některé Slaterovy determinanty tudíž nebudou v CISD metodě zahrnuty. (\textbf{obrázek?})
\end{itemize}

Kvůli těmto nedostatkům se dnes mnohem více používají metody, které si představíme v následujících dvou kapitolách.

\subsection{Mollerova--Plessetova metoda}


\begin{equation}
\hat{H}^0=\sum_i^N\hat{F}_i
\end{equation}



\subsection{Metody spřažených klastrů}
Metody spřažených klastrů (CC -- \textit{Coupled Cluster}) nesou od svého počátku českou stopu. Tyto metody totiž použil v kontextu kvantové chemie poprvé Jiří Čížek a později také Josef Paldus.

Klíčovým pojmem v této metodě jsou takzvané excitační operátory
\begin{eqnarray}
\hat{T}_1\psi_0=\sum^N_{a=1}\sum_{m=n+1}^\infty t_a^m\psi_a^m \\
\hat{T}_2\psi_0=\sum_{a=1}^N \sum_{b\neq a}^N\sum_{m=N+1}^\infty \sum_{n=N+1}^\infty t_{ab}^{mn}\psi_{ab}^{mn} \\
\hat{T}=\hat{T}_1+\hat{T}_2+\cdots \hat{T}_N ,
\end{eqnarray}
kde $\hat{T_1}$ je generátor mono--excitací, $\hat{T}_2$ je generátor double--excitací atp. a $t_x^x$ jsou číselné koeficienty. Tyto operátory tedy vytváří z jednoho Slaterova determinantu $\psi_0$ (vetšinou získaného metodou HF) sadu excitovaných Slaterových determinantů, s nimiž jsme se již potkali v kapitolce o metodě konfigurační interakce. V tomto případě ovšem vlnovou funkci zapisujeme v poněkud zvláštním tvaru
\begin{eqnarray}
\psi^{CC} = e^{\hat{T}} \psi_0 ,  \\
\end{eqnarray}
kde $\psi_0$ je referenční funkce, většinou z metody HF. Zde se poprvé setkáváme s podivným pojmem exponenciály operátoru. Není třeba se jej leknout, jen si musíme uvědomit, jak je vlastně definována normálni exponenciální funkce $e^x$. Jedna z možných definic je
\begin{equation}
e^x=\sum_{i=1}^\infty \frac{x^n}{n!} .
\end{equation}
K tomuto vzorečku lze dojít například aplikací Taylorova rozvoje na $e^x$.
Tu samou definici nyní můžeme snadno aplikovat i na operátory, musíme jen vědět, jak operátory umocňovat.
To jsme si ale řekli již v kapitole \ref{kap:OperaceSOperatory}. Operátor umocněný na $n$-tou prostě znamená, že jej aplikujeme $n$-krát za sebou na tu samou funkci.
Exponenciela excitačního operátoru tedy vyjde jako 
\begin{equation}
e^{\hat{T}} = 1+\hat{T}+\frac{\hat{T}^2}{2}+\cdots.
\end{equation}
Díky vlastnostem excitačního operátoru je tato řada konečná, neboť máme konečný počet elektronů, které můžeme excitovat.

Nyní bychom se mohli ptát, jaká je vlastně výhoda takto složitého zápisu.
Pokud použíjeme úplný excitační operátor, tak dojdeme k exaktnímu řešení stejně jako metoda FCI.
Výhody se ale objeví, když excitační operátor začneme ořezávat. Když například vezmeme v potaz pouze mono-- a double--excitace, dostaneme metodu CCSD -- \textit{Coupled Cluster Singles Doubles}).
Narozdíl od metody CISD je ovšem CCSD size-konzistentní a navíc dostaneme větší podíl korelační energie.
To je dáno právě speciálním tvarem CC vlnové funkce, díky kterému i na úrovni CCSD dostaneme například i částečný příspěvek kvadruple--excitací, jelikož v exponencielní řadě je přítomen operátor $\hat{T}_2^2$.

Vzoreček pro energii CC metod a jeho odvození je složitějši, je třeba řešit složité nelineární rovnice pro koeficienty $t$ excitačního operátoru. Metody spřažených klastrů jsou stejně jako poruchové metody size-konzistnentní, ale nejsou variační. Ukazuje se ale, že v praxi je právě size-konzistence důležitější, a proto se metody konfigurační interakce dnes používají méně často než poruchové metody a metody spřažených klastrů.

Jak jsme si již řekli v předchozí podkapitole, metody konfigurační interakce konvergují dosti pomalu k přesnému řešení. Tvar vlnové funkce v metodě spřažených klastrů tuto konvergenci značně urychluje. Už se zahrnutím trojitých excitací dostáváme výsledky s chemickou přesností (tzn. relativní energie s přesností 1\,kcal/mol což je zhruba 4\,kJ/mol.
Zlatým standardem kvantové je momentálně metoda CCSD(T), ve které jsou trojité excitace zahrnuty v rámci poruchového počtu. Výpočetní náročnost této metody dovoluje na současné úrovni výpočty molekul s 10-XX atomy. Užitím speciálních technik lineárního škálování lze ale toto použití značně rozšířit, nedávno tak byla tato metoda aplikována na malou molekulu proteinu obsahující stovky atomů. 

\subsection{Multireferenční metody}

Základní stav molekuly vodíku můžeme popsat následujicím Slaterovým determinantem:
\begin{equation}
\Psi (1,2)=\frac{1}{\sqrt{2}}
\begin{vmatrix}
\sigma_g(1)\alpha (1) & \sigma_g(2)\alpha (2) \\
\sigma_g(2)\beta (2) & \sigma_g(2)\beta (2)
\end{vmatrix}
=\frac{1}{\sqrt{2}}\sigma_g(1)\sigma_g(2)(\alpha (1)\beta (2)-\alpha (2)\beta (1))
\end{equation}

\subsection{Metody funkcionálu hustoty}

Pomocí metod založených na teorii funkcionálu hustoty (dále jen DFT metod) se v posledních zhruba dvaceti let provádí většina výpočtů elektronové struktury. Popularita DFT metod je dána jejich přijatelnou výpočetní náročností, která o mnoho nepřevyšuje Hartreeho-Fockovu 
metodu. Výpočty jsou ale typicky daleko přesnější, neboť DFT metody zahrnují korelační energii. Efektivita DFT je dána tím, že nepracuje s poměrně složitou vlnovou funkcí (tj. funkcí 3N souřadnic elektronů), ale s tzv. elektronovou hustotou, což je funkce pouze tří prostorových souřadnic. 

Představme si základní pojmy. Začněme pojmem \textbf{funkcionál}. Pojďme si nejprve připomenout, jak funguje funkce. Do funkce vložíme nezávislé proměnou, tedy nějaké číslo, a na oplátku dostaneme číslo jiné, neboli závisle proměnnou. Jde tedy o zobrazení z prostoru  (kupř. reálných) čísel opět do prostoru reálných čísel. U funkcionálu je to velmi podobné, akorát na vstupu není číslo, ale funkce. Pokud to tedy řekneme více matematicky, funkcionál je zobrazení z prostoru funkcí na prostor (kupř. reálných) čísel.

Takovým jednoduchým funkcionálem je určitý integrál
$$
F[f(x)] = \int_a^b f(x) \mathrm{d}x, 
$$
Určitý integrál potřebuje dodat vstupní funkci a po jeho vyčíslení dostaneme jedno jediné číslo. Povšimněte si zde zápisu funkcionálu pomocí hranatých závorek.  
Pro složitější příklad funkcionálu nemusíme chodit daleko, stačí se podívat, jak vypadá funkcionál energie v kvantové mechanice.
$$
E[\psi] = \int \psi^*\hat{H}\psi \mathrm{d}\tau .
\label{rov:dft:efunkc}
$$
S některými funkcionály jsme se tak již setkali.

Mezi funkcemi a funkcionály existují i další podobnosti. Pojmy jako minimum a maximum funkcionálu mají prakticky stejný význam. Existuje i funkcionální analogie k dobře známé derivaci funkce --- mluvíme o \textbf{variaci funkcionálu}.
Když děláme derivaci funkce, tak se vlastně díváme, co se děje se závisle proměnnou, když trochu změníme nezávisle proměnnou.
U variace je to podobné. Zajímá nás, jak se změní hodnota funkcionálu, když mírně změníme naši funkci. Přesná definice je složitější a neuvádíme ji zde. Bude ale pro nás důležité, že pro variace platí podobné vztahy, jako pro derivace. Dá se například ukázat, že v minimu funkcionálu je jeho variace nulová. Již dobře známý variační princip (konečně víme, proč se mu tak říká!) pak můžeme napsat jednoduše pomocí variace funkcionálu energie \eqref{rov:dft:efunkc} jako
$$
\delta E[\Psi] = 0 .
$$

Nyní se můžeme vrátit k definici ústřední veličiny této kapitoly --- elektronové hustoty $\rho(\mathbf{r})$.
Elektronová hustota má na rozdíl od vlnové funkce přímý fyzikální význam. Jedná se o pravděpodobnost, že v nějakém bodě prostoru najdeme \textbf{nějaký elektron}. Je důležité si uvědomit rozdíl mezi elektronovou hustotou a čtvercem vlnové funkce, který taktéž udává hustotu pravděpodobnosti. Čtverec vlnové funkce nám udává pravděpodobnost, že první elektron má spin $m_{s1}$ a je v bodě $\mathbf{r}_1$, druhý elektron má spin $m_{s2}$ a je v bodě $\mathbf{r}_2$ atd. Jedná se tedy o mnohem složitější veličinu, která závisí na celkem $4N$ proměnných, zatímco elektronová hustota závisí jen na třech proměnných.

Elektronová hustota souvisí s vlnovou funkcí systému dle vztahu
\begin{equation}
\rho=N \int |\psi(\textbf{r}_1,\textbf{r}_2,...,\textbf{r}_n)|^2 \mathrm{d}\textbf{r}_2\dots\mathrm{d}\textbf{r}_n .
\label{rov:dft:defrho}
\end{equation}
Integrujeme tedy čtverec vlnové funkce přes všechny elektrony kromě prvního. Jak již bylo řečeno, chceme pravděpodobnost najití jakéhokoli elektronu, poloha ostatních nás nezajímá, a musíme tedy přes ně prointegrovat. První elektron ale není nijak odlišný od ostatních, stejně dobře bychom ale mohli udělat to samé pro druhý elektron. Jelikož je vlnová funkce antisymetrická, došli bychom ke stejnému výsledku. Z toho vyplývá násobící faktor $N$ před integrálem.
Z této definice hned plyne několik důležitých vlastností.

\begin{itemize}
\item Elektronová hustota je nezáporná veličina, platí tedy
\begin{equation}
\rho(\mathbf{r})  > 0 .
\end{equation}
\item Pokud zintegrujeme elektronovou hustotu přes celý prostor, dostaneme počet elektronů v systému jako
\begin{equation}
\int \rho\mathrm{d}r = N
\end{equation}

\item V poloze jader má elektronová hustota maxima.
\item Pro tato maxima platí
%TOPETR: v poznamkach mas, ze to je nejaky teorem, ale neprectl jsem jaky....
\begin{equation}
\lim_{r_i \to 0} \left[ \frac{\delta}{\delta r}+2Z_A\right]\hat{\rho(r)}=0, 
\end{equation}
kde $\hat{\rho}$ je angulárně zprůměrovaná hodnota elektronové hustoty a $Z_A$ je náboj příslušného atomového jádra.
\end{itemize}

Z těchto vlastností vidíme, že pokud známe elektronovou hustotu, tak zároveň také můžeme zjistit počet elektronů, polohu jader i jejich náboj. To jsou ale přesně ty informace, které jsou potřeba ke specifikaci molekulárního hamiltoniánu
\begin{equation}
\hat{H}=\sum_{i=1}^N -\frac{1}{2}\Delta_i+\sum_{i=1}^N\sum_{j=i+1}^N\frac{1}{r_{ij}}-\sum_{i=1}^N\sum_{A=1}^K \frac{Z_A}{r_{iA}} ,
\label{rov:dft:molham}
\end{equation}
kde jsme vynechali pro jednoduchost člen popisující odpuzování jader, jenž je stejně v rámci Bornovy--Oppenheimerovy aproximace roven konstantě. Pokud ale elektronová hustota jednoznačně určuje hamiltonián, tak poté z řešení SCHR taky energii a vlnovou funkci, a tudíž všechny potřebné veličiny. Podobným způsobem se zřejmě ubíraly úvahy Hohenberga a Kohna, kteří postavili teorii DFT na pevné fyzikální základy.

\subsection{Hohenbergovy--Kohnovy teorémy}

V roce 1964 publikovali Hohenberg s Kohnem slavný článek, který odstartoval vývoj DFT metod.
V tomto článku byly dokázány dva důležité teorémy. K jejich lepšímu pochopení si ještě musíme definovat pojem \textbf{externího potenciálu} $\nu_{ext}$. Ten má pro případ hamiltoniánu \eqref{rov:dft:molham} tvar
\begin{equation}
\nu_{ext} = \sum_{i=1}^N\sum_{A=1}^K \frac{Z_A}{r_{iA}} .
\end{equation} 
Jedná se tedy o celkovou interakci elektronů s coulombickým potenciálem atomových jader. Jelikož se v DFT na vše díváme z pohledu elektronů, tak se tomu potenciálu říká externí, jelikož nepochází ze samotných elektronů. Je důležité si uvědomit, že externí potenciál vlastně definuje hamiltonián \eqref{rov:dft:molham}, neboť ostatní členy triviálně závisejí pouze na počtu elektronů v systému. Nyní již můžeme přejít ke slíbeným teorémům.

\textbf{První Hohenbergův--Kohnův teorém} hovoří o významnosti elektronové hustoty. Zní takto:

\bigskip
\noindent \uv{\textbf{Pro libovolný systém interagujících elektronů je externí potenciál $\nu_{ext}$ jednoznačně určen elektronovou hustotou} (až na konstantu)}

\bigskip
Důsledek tohoto tvrzení jsme si již naznačili dříve. Pokud máme jednoznačně daný externí potenciál, pak také známe hamiltonián a můžeme v principu spočítat vlnovou funkci i energii, které jsou tudíž jednoznačně určeny pouze elektronovou hustotou. \textbf{Elektronová energie je tedy jednoznačným funkcionálem elektronové hustoty} neboli v matematickém zápisu
\begin{equation}
E_{el}=E_{el}[\rho] .
\end{equation}
Tvar tohoto funkcionálu je nezávislý na externím potenciálu a je tedy univerzální pro všechny atomy a molekuly.
\textbf{ToPS:Z čeho plyne ta univerzalita?}
Pokud bychom tedy tento funkcionál znali a znali také správnou hustotu, tak bychom mohli získat energii i bez řešení Sch\"{o}dingerovy rovnice a hledání vlnových funkcí.

\bigskip
Pro důkaz tohoto teorému si musíme ještě odvodit pomocný vztah pro integraci externího potenciálu vynásobeného vlnovou funkcí.
Rozepišme si nejprve externí potenciál na součet jednoelektronových příspěvků:
\begin{equation}
\nu_{ext}=\sum_{i=1}^N \nu_i = \sum_{i=1}^N \sum_{A=1}^K \frac{Z_A}{r_{iA}}  
\label{rov:dft:nui}
\end{equation}
Pojďme si nyní vyčíslit integrál
\begin{equation}
\int \nu_{ext} \psi(\textbf{r}_1,...,\textbf{r}_N)^2\mathrm{d}\tau .
\end{equation}
Dosazením vztahu \eqref{rov:dft:nui} a vhodným přeuspořádáním mnohonásobného integrálu dostaneme
\begin{equation}
\sum_i^N \int \nu_i \left[\int \psi(\mathbf{r}_1,...,\mathbf{r}_N)\prod_{j\neq i}\mathrm{d}\textbf{r}_j\right] \mathrm{d}\textbf{r}_i .
\end{equation}
Výraz v hranaté závorce není ale nic jiného než elektronová hustota $\rho$ (až na násobný faktor $N$), dostáváme tedy finální vztah
\begin{equation}
\int \nu_{ext} \psi(\textbf{r}_1,...,\textbf{r}_N)^2\mathrm{d}\tau = \sum_{i=1}^N \int \nu_i(\textbf{r}_1)\frac{\rho(\textbf{r}_i)}{N}= \int \nu_i(\textbf{r})\rho(\textbf{r}) \mathrm{d}r .
\label{rov:dft:intnupsi}
\end{equation}

\bigskip 
\textbf{Důkaz HK1:} Důkaz se provádí sporem. Předpokládejme, že jedné dané hustotě $\rho$ přísluší dva různé externí potenciály $\nu_{ext1}$ a $\nu_{ext2}$, kterým přísluší hamiltoniány $\hat{H}_1$ a $\hat{H}_2$ a vlnové funkce $\psi_1$ a $\psi_2$.  Pak díky variačnímu principu musí platit

\begin{equation}
E_1 = \int \psi_1^* \hat{H}_1 \psi_1 \mathrm{d}\tau < \int \psi_2^* \hat{H}_1 \psi_2 . \mathrm{d}\tau ,
\end{equation}
protože když k hamiltoniánu $\hat{H_1}$ přiřadíme funkci $\psi_2$, která není jeho vlastní funkcí základního stavu, tak musíme dostat větší energii. Nyní chytře přepíšeme pravou stranu nerovnosti jako  
\begin{equation}
\int \psi_2^* \hat{H}_1 \psi_2 = \int \psi_2^* \hat{H}_2 \psi_2\mathrm{d}\tau  + \int \psi_2^* \left[\hat{H}_1-\hat{H}_2\right] \psi_2\mathrm{d}\tau = E_2 + \int \rho(\mathbf{r})(\nu_1-\nu_2)\mathrm{d}\mathbf{r}  
\end{equation}
V poslední rovnosti jsme využili toho, že oba hamiltoniány se liší pouze externím potenciálem, a použili jsme vztah \eqref{rov:dft:intnupsi}.

\noindent Vyšlo nám tedy, že platí nerovnost
\begin{equation}
E_1 < E_2+\int \rho(\mathbf{r})(\nu_1-\nu_2)\mathrm{d}\mathbf{r}
\label{rov:HK1_1}
\end{equation}
Tu samou argumentaci bychom ale mohli také aplikovat opačně a dostali bychom
\begin{equation}
E_2 < E_1+\int \rho(\mathbf{r})(\nu_2-\nu_1)\mathrm{d}\mathbf{r} .
\label{rov:HK1_2}
\end{equation}
Sečtením obou rovnic \eqref{rov:HK1_1} a \eqref{rov:HK1_2} tedy dostáváme
\begin{equation}
E_1 + E_2 < E_2 + E_1 ,
\end{equation}
což nemůže platit, a tedy náš původní předpoklad o existenci dvou různých externích potenciálů je také chybný.
\hfill {\footnotesize $\blacksquare$}

Formálně můžeme funkcionál energie napsat jako
\begin{equation}
E=\int \psi^*\hat{H}\psi \mathrm{d}\tau = \int \psi^*\hat{T}\psi\mathrm{d}\tau + \int \psi^*\hat{V_{el}}\psi\mathrm{d}\tau + \int \psi^*\nu_{ext}\psi\mathrm{d}\tau=\hat{T}[\rho]+\hat{V}_{ee}[\rho]+\hat{V}_{ne}[\rho] .
\end{equation}
Vidíme, že energie se stejně jako příslušný hamiltonián \eqref{rov:dft:molham} skládá ze tří různých příspěvků. Funkcionál energie tedy můžeme rozdělit na funkcionál kinetické energie elektronů $\hat{T}[\rho]$, funkcionál repulze elektronů $\hat{V}_{ee}[\rho]$ a funkcionál interakce elektronů s jádry (obecně s externím potenciálem) $\hat{V}_{ne}[\rho]$. 
Explicitní tvar posledně jmenovaného funkcionálu jsme si již vlastně odvodili rovnicí \eqref{rov:dft:intnupsi} a platí tedy
\begin{equation}
\hat{V}_{ne}[\rho] = \int \nu_i(\textbf{r})\rho(\textbf{r}) \mathrm{d}r .
\end{equation}
Součet zbylých dvou funkcionál nazýváme Hohenbergovým--Kohnovým funkcionálem a značíme $F_{HK}[\rho]$. 
Celkový funkcionál energie tedy můžeme zapsat jako
\begin{equation}
E[\rho] = \hat{T}[\rho]+\hat{V}_{ee}[\rho]+\hat{V}_{ne}[\rho] = F_{HK}[\rho] + \int \nu_i(\textbf{r})\rho(\textbf{r}) \mathrm{d}r .
\end{equation}

Přesný tvar Hohenbergova--Kohnova funkcionálu $F_{HK}[\rho]$ bohužel není dodnes znám. 
Ovšem i kdybychom jej znali, tak nám stále něco schází k jeho úspěšného použití. Nevíme totiž, jak získat pro daný systém správnou elektronovou hustotu $\rho$, kterou bychom do něj mohli dosadit. Samozřejmě kdybychom znali vlnovou funkci, tak stačí dosadit do definičního vztahu \eqref{rov:dft:defrho}. Tomu se ale právě chceme vyhnout! Smyslem celé teorie funkcionálu hustoty je vyhnout se přímému řešení elektronové SCHR.

Cestu k elektronové hustotě nám dává druhý teorém od Hohenberga a Kohna, který zní takto:

\bigskip
\textbf{Předpokládejme, že danému externímu potenciálu $\nu_{ext}$ přísluší elektronová hustota $\rho_0$. Pak pro jakoukoli jinou elektronovou hustotu\footnote{Přesněji řečeno, daná elektronová hustota musí být takzvaně $\nu$--{reprezentovatelná}, neboli musí k ní příslušet nějaký externí potenciál. Pro funkce, které toto nesplňují, 2. HK teorém neplatí.} $\rho^{\prime}$ bude platit:}
\begin{equation}
E[\rho_0] < E[\rho^{\prime}]
\label{rov:dft:HK2}
\end{equation}

Nejedná se o nic jiného než variantu variačního principu, který taky využijeme k důkazu.

\bigskip
\textbf{Důkaz:} Z prvního HK teorému plyne, že jakákoli funkce $\rho^{\prime}$ patří k externí potenciálu $\nu_{ext}^{prime}$, který je odlišný od $v_{ext}$), a přísluší k němu vlnová funkci $\psi^{prime}$. Pokud ale pro tuto vlnovou funkci vyčíslíme energii, pak nám z již známého variačního principu plyne
\begin{equation}
\int \psi^{\prime *} \hat{H} \psi^{\prime} \mathrm{d}\tau = E[\rho] > E [\rho_0],
\end{equation}
což jsme chtěli dokázat. \hfill {\footnotesize $\blacksquare$}

Druhý HK teorém nám tedy dává principiální návod, jak hledat elektronovou hustotu. Budeme hledat přes všechny možné hustoty a správná bude ta, která nám dá nejnižší energii.

Hohenbergovy--Kohnovy teorémy dávají DFT solidní fyzikální základ, moc nás ale neposunují k praktické aplikaci, jelikož neznáme přesný funkcionál $F_{HK}[\rho]$.
Praktickou cestu k DFT výpočtům ukázali až o pár let později Kohn s Shamem.

\subsection{Kohnovy--Shamovy rovnice}

Nyní stojíme před zásadním problémem najití alespoň přibližného funkcionálu $F_{HK}[\rho]$, ve kterém je zahrnuta kinetická energie elektronů, klasická Coulombická interakce mezi elektrony a dále korelační a výměnné efekty. 
%Obecně bychom chtěli spočítat co největší část energie pomocí známých dobře definovaných přibližných vztahů a zbylé části poté můžeme modelovat a odchylky poté můžeme modelovat třeba semiempiricky.(\footnote{Tento přístup by se dal přirovnat k situaci v chemické termodynamice, ve které počítáme se vzorečky platnými pro ideální chování a odchylky schováváme do aktivitních koeficientů).
Ukázalo se, že největší potíže činí dostatečně přesné vyjádření funkcionálu pro kinetickou energii.
Tento problém je tak zásadní, že budeme muset částečně obětovat náš původní cíl a vrátit se k molekulovým orbitalům.
Pokud totiž máme molekulové orbitaly, tak pomocí nich můžeme vyčíslit kinetickou energii dle vztahu
\begin{equation}
E_{kin}=\sum_{i=1}^N \int \varphi\frac{1}{2} \Delta_i \varphi \mathrm{d}\textbf{r}_i .
\end{equation}
S takto postavenou teorií přišli v roce 1965 Kohn s Shamem.

Obecná strategie odvození Kohnovy--Shamovy proceduru je následující. Definuje se fiktivní systém neinteragujících elektronů (podobně jako v Hartreeho--Fockově teorii zde elektrony interagují pouze skrze efektivní potenciál), který je zvolen tak, aby jeho elektronová hustota bylo rovna elektronové hustotě reálného systému. 
Funkcionál energie se rozepíše následujícím způsobem:
\begin{equation}
E[\rho]= T_{n}[\rho] + \frac{1}{2}\int \int \frac{\rho(\textbf{r}_1)\rho(\textbf{r}_2)}{r_{12}} + V_{nekl}(\rho) +\left[T[\rho]-T_{n}[\rho]\right] + \int \nu \rho \mathrm{d}\textbf{r},
\label{rov:dft:KSfunkc}
\end{equation}
kde $T_{n}[\rho]$ kinetická energie neinteragujícího systému, druhý člen odpovídá klasické mezi-elektronové repulzi (násobí se jednou polovinou, aby se interakce nezapočítávaly dvakrát), $V_{nekl}$ zahrnuje korelační a výměnnou energii elektronů, člen v hranaté závorce je rozdíl kinetických energií reálného a neinteragujícího systému  a poslední člen odpovídá interakci elektronů s jádry. Pokud všechny neznámé členy dáme dohromady dostaneme tzv. korelačně--výměnný potenciál
\begin{equation}	
E_{XC}[\rho]=\left[T[\rho]-T_{n}[\rho]\right] +  V_{nekl} .
\label{rov:dft:exc}
\end{equation}
% V_{ee}(\rho) - \frac{1}{2}\int \int \frac{\rho(\textbf{r}_1)\rho}{\textbf{r_{12}}}
Pokud na funkcionál \eqref{rov:dft:KSfunkc} nyní aplikujeme variační princip, tak dostaneme rovnice, které mají stejný tvar jako rovnice
pro systém neinteragujících elektronů. Jenže pro tento systém známe řešení! Stačí vyřešit vyřešit příslušnou jedno--elektronovou  	 Schr\"{o}dingerovu rovnici, velmi podobnou rovnicím Hartreeho--Focka
\begin{equation}
\left(-\frac{1}{2}\Delta_i + V_{eff} \right) \varphi_i =\epsilon_i \varphi_i ,
\label{rov:dft:KSeq}
\end{equation}
kde $V_{eff}$ je efektivní potenciál, pro který platí
\begin{equation}
V_{eff}=\nu_{ext}+\frac{1}{2}\frac{\rho(\textbf{r}^{\prime})}{|\textbf{r}-\textbf{r}^{\prime}|}\mathrm{d}\textbf{r}^{\prime}+u_{xc} ,
\end{equation}
kde $u_{xc}$ je funkcionální derivace (variace) výměnně--korelačního funkcionálu \ref{rov:dft:exc} tvar tohoto potenciálu byl odvozen tak, aby byly elektronové hustoty reálného i fiktivního systému stejné.
Rovnice \ref{rov:dft:KSeq} se nazývají Kohnovy--Shamovy a řeší se podobně jako rovnice Hartreeho--Fockovy rozvojem do báze AO.
Získáme tak sadu molekulových orbitalů, ze kterých pak dostaneme elektronovou hustotu dle vztahu (uvádíme bez důkazu)
\begin{equation}
\rho(\textbf{r}) = \sum_{i=1}^N |\varphi|_i^2
\label{rov:dft:KSrho}
\end{equation}
%Jelikož neinteragující systém byl definován tak, aby 
Tuto hustotu pak můžeme dosadit do funkcionálu \eqref{rov:dft:KSfunkc} a získáme tak požadovanou energii.

Kohn--Shamův přístup je v principu přesný, pokud bychom znali přesný tvar výměnně--korelačního funkcionálu $E_{XC}[\rho]$.
Ten sice neznáme, ale existuje spousta vztahů přibližných, o kterých pojednávají další kapitoly.


\subsection{Aproximace lokální hustoty}

LDA (Local density Approximation)
\begin{equation}
E_{xc}^{LDA}=\int \rho(\textbf{r})V_{xc}(\rho(\textbf{r}))\mathrm{d}\textbf{r} 
\end{equation}


\begin{equation}
V_{xc}(\rho)=V_x(\rho)+V_c(\rho)
\end{equation}
Pro výměnný člen platí pro homogenní plyn následující vztah:
\begin{equation}
V_x(\rho)=-\frac{3}{4}\left(\frac{3}{\pi}\right)^{\frac{1}{3}}\rho^{\frac{1}{3}}
\end{equation}
Pro korelační energii UEG nelze získat analytický výraz. Příslušné výpočty lze ale provést numericky a výsledek poté nafitovat. Výsledný korelační funkcionál je znám jako VWN (dle pánů Vosko-Wilk-Nusair).

LSDA....$E_{xc}=E_{xc}[\rho_\alpha\rho_\beta]$ 


\subsection{GGA a hybridní funkcionály}
GGA 
\begin{equation}
E_{xc}^{GGA}=\int \rho(\textbf{r})f(\rho,\Delta\rho\mathrm{d}\textbf{r} 
\end{equation}

Becke: B88, LYP (Lee-Yang-Parr),
kombinace: BP86, BLYP, PBE

meta-GGA

\textbf{Hybridní funkcionály}
\begin{equation}
E_X^{exact}=-\frac{1}{2}\sum_{i=1}^N\sum_{j=1}^N K_{ij}
\end{equation}

B3LYP:
\begin{equation}
E_{xc}^{B3LYP}=(1-a_0-a_x)E_x^{LDA}+a_0E_x^{exact}+a_xE_x^{B88}+(1-a_c)E_c^{VWN}+a_c E_c^{LYP}
\end{equation}
$a_0=0,2$; $a_x=0,72$; $a_c=0,81$

PBE0 (25\,\%), BMK,BHandHLYP(50\,\%)

\textbf{Long-range corrected funkcionály}
Mělo by platit, že $lim_{V_x\to \infty}=\frac{1}{r}$

\textbf{Disperzní korekce}

Grimmeho empirická korekce
\begin{equation}
E_{vdw}=-s_6\sum c^{ij}r_{ij}^{-6}
\end{equation}
Koeficient $s_6$ závisí na použitém funkcionálu, zatímco koeficienty $c_{ij}$ závisí na typu interagujících atomů.

\textbf{Dvojitě hybridní funkcionály}
S.Grimme 
\begin{equation}
E_{xc}^{hybrid}=a_1E_x^{GGA}+(1-a_1)E_x^{EXACT}+a_2E_c^{GGA}
\end{equation}
Pro tento tvar se vyřeší KS rovnice a získají KS orbitaly. Z těchto orbitalů se poté vypočítá MP2 korekce
ze vzorečku XX a přídá se k $E_{xc}^{hybrid}$

\begin{equation}
E_{xc}^{DH}=E_{xc}^{hybrid}+(1-a_2)E_c^{KS-MP2}
\end{equation}

Příkladem může být B2LYP s parametry $a_1=0,47$ a $a_2$=0,73

\textbf{Jakobův žebřík, obrázek (Genesis 28:10-12)}

