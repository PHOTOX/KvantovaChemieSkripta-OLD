%% Je třeba zkontrolovat rovnice a hlavně značení proměných. Je třeba lépe definovat co je r, R atp.

Obecná molekula může konat translační, vibrační nebo rotační pohyby. V~této kapitole si ukážeme pouze nejjednodušší příklad, totiž popis vibrace a rotace dvouatomové molekuly.

\noindent V~minulé kapitole jsme si ukázali Schrödingerovu rovnici pro jádra (\ref{rov:mol-jadschr})

\begin{displaymath}
\left(\hat{T}_R+E_{el}^{(j)}\right)\chi_j = E \chi_j,
\end{displaymath}

\noindent kde $E_{el}$ je elektronová energie jako funkce souřadnic atomového jádra, $\chi$ je vlnová funkce popisující pohyb atomových jader (tedy translaci, vibraci a~rotaci molekuly) a $E$ je celková energie molekuly zahrnující energii pohybu elektronu i jader.  Písmenem $j$ indexujeme elektronový stav.

\noindent Pro zjednodušení budeme v~následujícím textu uvažovat platnost Bornovy-Oppenheimerovy aproximace a dále pouze základní elektronový stav molekuly

\begin{displaymath}
\left(\hat{T}_R+E_{el}\right)\chi_j = E \chi_j.
\end{displaymath}
 
\noindent Jelikož uvažujeme dvouatomovou molekulu, bude elektronová energie $E_{el}$ pouze funkcí vzdálenosti dvou jader (označme je jako $\alpha$ a $\beta$) a tudíž nezávislá na jejich orientaci:

\begin{displaymath}
	E_{el}=E_{el} (\vert \textbf{R}_{\alpha}-\textbf{R}_{\beta} \vert ) = E_{el}(r).
\end{displaymath} 

Dalším krokem při řešení Schr\"odingerovy rovnice pro pohyb jader je rozdělení kinetické energie na kinetickou energii těžiště molekuly a kinetickou energii odpovídající relativnímu pohybu molekuly. Celý postup je popsán v~kapitole o~problému dvou částic. Zde si uvedeme pouze dvě konečné rovnice, ke kterým bychom se dostali

\begin{eqnarray}
\hat{H}_{tr}\psi_{tr}(R)=E_{tr}\psi_{tr}(R)
\label{vibrot:trans}\\
\hat{H}_{int}\psi_{int}(r)=E_{int}\psi_{int}(r)
\label{vibrot:vibrot}\\
E=E_{tr}+E_{int},
\end{eqnarray}

\noindent přičemž první rovnice \eqref{vibrot:trans} popisuje translaci molekuly jako celku (R je zde poloha těžiště) a druhá \eqref{vibrot:vibrot} její relativní pohyb, tedy vibraci nebo rotaci. Translaci jsme se  věnovali v~kapitole~\ref{kap:zakladniulohy}, nyní se v~dalším výkladu budeme věnovat vibracím a rotacím. Rovnici \ref{vibrot:vibrot} představující vnitřní problém si můžeme rozepsat následovně

\begin{equation}
\left[-\frac{\hbar^2}{2\mu}\Delta_{\textbf{r}}+E_{el}(r)\right]\psi_{int}=E_{int}\psi_{int}
\label{vibrot:vibrot2}
\end{equation}

Tato rovnice představuje molekulu v~centrálním potenciálu. Také centrálnímu potenciálu jsme se věnovali v~kapitole s~problémem dvou částic. Ukázali jsme si, že řešení hledáme ve tvaru součinu sférické harmonické funkce a radiální vlnové funkce

\begin{equation}
\psi_{int}(r,\theta,\psi)=R(r)Y^m_l(\theta,\psi).
\end{equation}

\noindent Takovouto vlnovou funkci můžeme dosadit do Schr\"odingerovy rovnice a posléze vydělit sférickými harmonickými funkcemi, abychom dostali radiální Schr\"odingerovu rovnici

\begin{equation}
-\frac{\hbar^2}{2\mu}\left(R''+\frac{2}{r}R'\right)+\frac{l\left(l+1\right)\hbar^2}{2\mu r^2}R+\hat{V}R=E_{int}R,
\end{equation}

\noindent kterou si dále zjednodušíme zavedením substituce
\begin{displaymath}
F(r)=rR(r)
\end{displaymath}

\noindent na

\begin{equation}
-\frac{\hbar^2}{2\mu}F''+\left[E_{el}+\frac{l\left(l+1\right)\hbar^2}{2\mu r^2}\right]F=E_{int}F.
\label{vibrot:vibrot3}
\end{equation}

\noindent Poslední rovnici je možné zjednodušit, zavedeme-li si dvě aproximace. 

\begin{itemize}

\item Nejprve použijeme Taylorův rozvoj do druhého řádu pro vyjádření elektronové energie

\begin{equation}
E_{el}\approx E_{el}(R_{eq})+\left(\frac{dE_{el}}{dr}\right)(r-R_{eq})+\frac{1}{2}\left(\frac{d^2E_{el}}{dr^2}\right)(r-R_{eq})^2, 
\end{equation}

\noindent kde $R_{eq}$ je rovnovážná mezijaderná vzdálenost molekuly. První člen určuje pouze relativní hladinu energie, a proto si můžeme zvolit $E_{el}(R_{eq})=0$. Druhý člen je nulový, protože se nacházíme v~minimu (pro $R = R_{eq}$) a teprve třetí člen bude nenulový. Zavedeme-li si substituci

\begin{eqnarray}
k=\left(\frac{d^2E_{el}}{dr^2}\right)\nonumber \\
x=r-R_{eq}\nonumber,
\end{eqnarray}

\noindent můžeme Taylorův rozvoj energie přepsat do následujícího tvaru

\begin{equation}
E_{el}\approx\frac{1}{2}\left(\frac{d^2E_{el}}{dr^2}\right)(r-R_{eq})^2=\frac{1}{2}kx^2.
\end{equation}

\noindent Vidíme tedy, že jsme složitý mezimolekulový potenciál nahradili parabolou. Aproximujeme chemickou vazbu modelem harmonického oscilátoru.

\item Dále uděláme Taylorův rozvoj ještě pro druhý člen (odpovídající rotaci) v~závorce rovnice \ref{vibrot:vibrot3}

\begin{equation}
\left(\frac{l\left(l+1\right)\hbar^2}{2\mu}\right)\frac{1}{r^2} =
\left(\frac{l\left(l+1\right)\hbar^2}{2\mu}\right)
\left[\frac{1}{R_{eq}^2}
-\frac{2}{R_{eq}^3}(r-R_{eq})
+\frac{3}{R_{eq}^4}(r-R_{eq})^2+...
\right],
\end{equation}
ze kterého budeme v~tomto případě uvažovat pouze nultý člen
\begin{equation}
\left(\frac{l\left(l+1\right)\hbar^2}{2\mu}\right)\frac{1}{r^2} \approx \left(\frac{l\left(l+1\right)\hbar^2}{2\mu}\right)\frac{1}{R_{eq}^2}
\end{equation}
\end{itemize}
\noindent To se může zdát jako dosti drastická aproximace. Fyzikálně to znamená, že během rotačního pohybu máme fixní vzdálenosti jader a tudíž zanedbáváme vliv vibrace. Tomuto se říká aproximace tuhého rotoru.


Uvažujeme-li tyto dvě aproximace, rovnice \ref{vibrot:vibrot3} přejde na 

\begin{equation}
-\frac{\hbar^2}{2\mu}F''+\left[\frac{1}{2}kx^2+\frac{l\left(l+1\right)\hbar^2}{2\mu R_{eq}^2}\right]F=E_{int}F 
\label{vibrot:vibrot4}
\end{equation}

\noindent a po změně pořadí členů na 

\begin{equation}
-\frac{\hbar^2}{2\mu}F''+\frac{1}{2}kx^2F=\left(E_{int}-\frac{l\left(l+1\right)\hbar^2}{2\mu R_{eq}^2}\right)F. 
\label{vibrot:vibrot5}
\end{equation}

\noindent Poslední rovnice je rovnicí pro jednodimenzionální harmonický oscilátor. Přepišme si ji ještě zavedením

\begin{displaymath}
E'=E_{int}-\frac{l\left(l+1\right)\hbar^2}{2\mu R_{eq}^2}
\end{displaymath}

\noindent na

\begin{equation}
-\frac{\hbar^2}{2\mu}F''+\frac{1}{2}kx^2F=E'F. 
\label{vibrot:vibrot6}
\end{equation}

Energii harmonického oscilátoru známe a můžeme tedy dále psát

\begin{equation}
E'=\left(n+\frac{1}{2}\right)h\nu,
\end{equation}

\noindent kde $\nu=\frac{1}{2\pi}\sqrt[]{\frac{k}{\mu}}$. Při uvažování těchto vztahů dostaneme konečný vztah pro vnitřní energii, tedy

\begin{equation}
\boxed{E_{int}\approx \left(n+\frac{1}{2}\right) h\nu+\frac{\hbar^2l(l+1)}{2\mu R_{eq}^2}.}
\end{equation}

\noindent Dva členy v~poslední rovnici zastupují harmonický oscilátor (vibrace) a tuhý rotor (rotace). 

\noindent Aproximace harmonického oscilátoru a tuhého rotoru mají své limity. Harmonický oscilátor nepopisuje disociaci molekuly a má ekvidistantní vibrační hladiny, kdežto z experimentálních vibračních spekter víme, že tomu tak u skutečných molekul není. Místo harmonického oscilátoru je možné použít oscilátoru Morseova

\begin{equation}
E_{el}=D_e\left[1-e^{a(r-R_{eq})}\right],
\end{equation}

\noindent který už uvažuje s~disociací vazby (disociační energie je $D_e$ a parametr $a$ se vztahuje k~šířce potenciálu, $a=\sqrt[]{\frac{k}{2D_e}}$).
\noindent Pokud bychom chtěli vibrace a rotace popsat ještě přesněji, bylo by třeba uvažovat Coriolisovy interakce, tedy skutečnost, že vibrační a rotační pohyb není nezávislý.

% Zde by ještě chtělo krátkou diskusi výběrových pravidel.

% Z pohledu spektroskopie nám nestačí znát pouze energii jednotlivých hladin, ale také pravděpodobnost přechodu mezi nimi. Zatímce energetické rozdíly určují pozici píku, pravděpodobnost přechodu určuje jeho výšku. 

% Z časově závislé poruchové teorie (kterou se v těchto skriptech nezabýváme) vyplývá, že pravděpodobnost světlem iniciovaného přechodu mezi dvěma stavy je úměrná maticovým elementům:
%\begin{equation}
%XXX
%\end{equation}
%Výpočtem těchto integrálů pak lze odvodit výběrová pravidla, která platí pro daný typ spektroskopie. Například dosazením vlnových funkci HO a výpočtem integrálu se dá odvodit výběrové pravidlo pro IR spektroskopii:
%\begin{equation}
%\Delta n = \pm 1
%\end{equation}  
%Pro rotační spektroskopii platí:
%\begin{equation}
%\delta l = 
%\end{equation}

%Je třeba si ale uvědomit, že dané výběrové pravidla platí pouze na úrovni dané aproximace. Pokud bychom například Taylorův rozvoj "XX" utli až po čtvrtém členu, byly by povoleny i přechody ($\Delta n =\pm 2). Těmto přechodům se říká overtonové a běžně je ve spektrech můžeme najít, i když s nižší intenzitou.

